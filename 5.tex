\chapter{Riemann 积分}

%——————————————————————————————————%

\section{函数的积分}

\begin{definition}
    闭区间 $[a,b]$ 的一个分割 $P$,指在 $a$ 在 $b$ 之间插入的有限个分点
    \[P : a = x_0 < x_1 < \cdots < x_n = b\]
    这些分点将 $[a,b]$ 分成 $m$ 个闭子区间
    \[[x_0,x_1],[x_1,x_2],\cdots,[x_{n -1},x_n]\]
    其中第 $i$ 个闭子区间的长度为
    \[\Delta x_i = x_i - x_{i - 1}\]
    我们把
    \[|P| = \max\{\Delta x_1,\Delta x_2,\cdots,\Delta x_n\}\]
    称为分割 $P$ 的模。
\end{definition}

\begin{definition}
    若在 $[a,b]$ 的一个分割 $P$ 的每个闭区间 $[x_{i - 1}, x_i]$ 上都选定一个标记点 $\xi_i \in [x_{i -1}, x_i](i = 1,2,\cdots,n)$,则称标记点 $\xi_1,\xi_2,\cdots,\xi_n$ 给出闭区间 $[a,b]$ 的一个标记分割 $(P,\xi)$。

    其中把这些标记点的集合记为 $\xi$。
\end{definition}

\begin{definition}{Riemann 积分}
    设函数 $f$ 在闭区间 $[a,b]$ 上有定义。若 $I \in \mathbb{R}$ 使得对 $\forall \varepsilon > 0$,$\exists \delta > 0$,只要闭区间 $[a,b]$ 的分割 $P$ 满足 $|P| < \delta$,而不管如何选择 $\xi \in [x_{i - 1}, x_i](1 \leqslant i \leqslant n)$,都有
    \[|I - \sum_{i =1}^{n}f(\xi_i)\Delta x_i| < \varepsilon\]
    则称 $f$ 在 $[a,b]$ 上 Riemann 可积,称 $I$ 为 $f$ 在 $[a,b]$ 上的 Riemann 积分。即
    \[I = \lim_{|P| \to 0}\sum_{i = 1}^{n}f(\xi_i)\Delta x_i\]
    称和式
    \[\sum_{i = 1}^{n}f(\xi_i)\Delta x_i\]
    为 $f$ 的 Riemann 和(或积分和)。
    函数 $f(x)$ 在 $[a,b]$ 上的积分记为
    \[\int_a^b f(x)\mathrm{d}x\]
    其中 $a,b$ 分别称为积分下限和积分上限,$f$ 称为被积函数,$f(x)\mathrm{d}x$ 称为被积表达式,$x$ 称为积分变量。从而
    \[\int_a^b f(x)\mathrm{d}x := \lim_{|P| \to 0}\sum_{i = 1}^{n}f(\xi_i)\Delta x_i\]
\end{definition}

\begin{theorem}{Riemann 积分的性质}
    \begin{enumerate}
        \item 设 $f$ 在 $[a,b]$ 上可积且非负,则
              \[\int_a^b f(x)\mathrm{d}x > 0\]
        \item 设 $f$ 和 $g$ 在 $[a,b]$ 上可积,且 $f \geqslant g$ 在 $[a,b]$ 上成立,则
              \[\int_a^b f(x)\mathrm{d}x \geqslant \int_a^b g(x)\mathrm{d}x\]
        \item 若 $f$ 和 $g$ 在 $[a,b]$ 上可积,则 $f \pm g$ 在 $[a,b]$ 上也可积,且
              \[\int_a^b(f(x)\pm g(x))\mathrm{d}x = \int_a^b f(x)\mathrm{d}x \pm \int_a^b g(x)\mathrm{d}x\]
        \item 若 $f$ 在 $[a,b]$ 上可积,则对 $\forall c$,$cf$ 在 $[a,b]$ 上也可积,且
              \[\int_a^b cf(x)\mathrm{d}x = c\int_{a}^{b} f(x)\mathrm{d}x\]
    \end{enumerate}
\end{theorem}

\hfill

\begin{example}
    \[\int_{a}^{b} c \mathrm{d}x = c(b - a)\]
\end{example}

\hfill

\begin{example}
    \[\int_{a}^{b} x \mathrm{d}x = \frac{b^2 - a^2}{2}\]
\end{example}

\begin{theorem}{Newton-Leibniz 公式}
    设函数 $f$ 在 $[a,b]$ 上可积,且在 $(a,b)$ 上有原函数 $F$。若 $F$ 在 $[a,b]$ 是连续,则必有
    \[\int_{a}^{b} f(x) \mathrm{d}x = F(b) - F(a)\]
\end{theorem}
令
\[F(x)\bigg|_a^b := F(b) - F(a)\]
则
\[\int_a^b f(x)\mathrm{d}x = F(x)\bigg|_a^b = \int f(x)\mathrm{d}x\bigg|_a^b\]
注意到
\[\mathrm{d}F(x) = f(x)\mathrm{d}x\]
从而有
\[\int_a^b \mathrm{d}F(x) = F(x)\bigg|_a^b\]

%——————————————————————————————————%

\section{可积函数的性质}


%——————————————————————————————————%

\section{微积分基本定理}






%——————————————————————————————————%

\section{分部积分与换元}






%——————————————————————————————————%

\section{可积性理论}






%——————————————————————————————————%

\section{Lebesgue 定理}






%——————————————————————————————————%

\section{反常积分}






%——————————————————————————————————%

\section{数值积分}






%——————————————————————————————————%