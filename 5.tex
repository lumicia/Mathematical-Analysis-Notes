\chapter{Riemann 积分}

%——————————————————————————————————%

\section{函数的积分}

\begin{definition}
    闭区间 $[a, b]$ 的一个分割 $P$,指在 $a$ 在 $b$ 之间插入的有限个分点
    \[P : a = x_0 < x_1 < \cdots < x_n = b\]
    这些分点将 $[a, b]$ 分成 $m$ 个闭子区间
    \[[x_0,x_1],[x_1,x_2],\cdots,[x_{n -1},x_n]\]
    其中第 $i$ 个闭子区间的长度为
    \[\Delta x_i = x_i - x_{i - 1}\]
    我们把
    \[|P| = \max\{\Delta x_1,\Delta x_2,\cdots,\Delta x_n\}\]
    称为分割 $P$ 的模。
\end{definition}

\begin{definition}
    若在 $[a, b]$ 的一个分割 $P$ 的每个闭区间 $[x_{i - 1}, x_i]$ 上都选定一个标记点 $\xi_i \in [x_{i -1}, x_i](i = 1,2,\cdots,n)$,则称标记点 $\xi_1,\xi_2,\cdots,\xi_n$ 给出闭区间 $[a, b]$ 的一个标记分割 $(P,\xi)$。

    其中把这些标记点的集合记为 $\xi$。
\end{definition}

\begin{definition}{Riemann 积分}
    设函数 $f$ 在闭区间 $[a, b]$ 上有定义。若 $I \in \mathbb{R}$ 使得对 $\forall \varepsilon > 0$,$\exists \delta > 0$,只要闭区间 $[a, b]$ 的分割 $P$ 满足 $|P| < \delta$,而不管如何选择 $\xi \in [x_{i - 1}, x_i](1 \leqslant i \leqslant n)$,都有
    \[|I - \sum_{i =1}^{n}f(\xi_i)\Delta x_i| < \varepsilon\]
    则称 $f$ 在 $[a, b]$ 上 Riemann 可积,称 $I$ 为 $f$ 在 $[a, b]$ 上的 Riemann 积分。即
    \[I = \lim_{|P| \to 0}\sum_{i = 1}^{n}f(\xi_i)\Delta x_i\]
    称和式
    \[\sum_{i = 1}^{n}f(\xi_i)\Delta x_i\]
    为 $f$ 的 Riemann 和(或积分和)。
    函数 $f(x)$ 在 $[a, b]$ 上的积分记为
    \[\int_a^b f(x)\mathrm{d}x\]
    其中 $a, b$ 分别称为积分下限和积分上限,$f$ 称为被积函数,$f(x)\mathrm{d}x$ 称为被积表达式,$x$ 称为积分变量。从而
    \[\int_a^b f(x)\mathrm{d}x := \lim_{|P| \to 0}\sum_{i = 1}^{n}f(\xi_i)\Delta x_i\]
\end{definition}

\begin{theorem}{Riemann 积分的性质}
    \begin{enumerate}
        \item 设 $f$ 在 $[a, b]$ 上可积且非负,则
              \[\int_a^b f(x)\mathrm{d}x > 0\]
        \item 设 $f$ 和 $g$ 在 $[a, b]$ 上可积,且 $f \geqslant g$ 在 $[a, b]$ 上成立,则
              \[\int_a^b f(x)\mathrm{d}x \geqslant \int_a^b g(x)\mathrm{d}x\]
        \item 若 $f$ 和 $g$ 在 $[a, b]$ 上可积,则 $f \pm g$ 在 $[a, b]$ 上也可积,且
              \[\int_a^b(f(x)\pm g(x))\mathrm{d}x = \int_a^b f(x)\mathrm{d}x \pm \int_a^b g(x)\mathrm{d}x\]
        \item 若 $f$ 在 $[a, b]$ 上可积,则对 $\forall c$,$cf$ 在 $[a, b]$ 上也可积,且
              \[\int_a^b cf(x)\mathrm{d}x = c\int_{a}^{b} f(x)\mathrm{d}x\]
    \end{enumerate}
\end{theorem}

\hfill

\begin{example}
    \[\int_{a}^{b} c \mathrm{d}x = c(b - a)\]
\end{example}

\hfill

\begin{example}
    \[\int_{a}^{b} x \mathrm{d}x = \frac{b^2 - a^2}{2}\]
\end{example}

\begin{theorem}{Newton-Leibniz 公式}
    设函数 $f$ 在 $[a, b]$ 上可积,且在 $(a, b)$ 上有原函数 $F$。若 $F$ 在 $[a, b]$ 是连续,则必有
    \[\int_{a}^{b} f(x) \mathrm{d}x = F(b) - F(a)\]
\end{theorem}
令
\[F(x)\bigg|_a^b := F(b) - F(a)\]
则
\[\int_a^b f(x)\mathrm{d}x = F(x)\bigg|_a^b = \int f(x)\mathrm{d}x\bigg|_a^b\]
注意到
\[\mathrm{d}F(x) = f(x)\mathrm{d}x\]
从而有
\[\int_a^b \mathrm{d}F(x) = F(x)\bigg|_a^b\]

%——————————————————————————————————%

\section{可积函数的性质}

\begin{theorem}
    设函数 $f$ 在 $[a, b]$ 上可积,则 $f$ 在 $[a, b]$ 上有界。
\end{theorem}

\begin{theorem}{积分的可加性}
    设 $c \in (a, b)$,函数 $f$ 在 $[a,c],[c, b]$ 上可积,则 $f$ 在 $[a, b]$ 也上可积,且
    \[\int_{a}^{b} f(x) \mathrm{d}x = \int_{a}^{c} f(x) \mathrm{d}x + \int_{c}^{b} f(x) \mathrm{d}x\]
\end{theorem}

\begin{theorem}
    若 $f$ 在 $[a, b]$ 上连续且非负,但 $f$ 不恒等于 $0$,则
    \[\int_{a}^{b} f(x) \mathrm{d}x > 0\]
\end{theorem}

\begin{theorem}
    设函数 $f$ 在 $[a, b]$ 上可积,则 $|f|$ 在 $[a, b]$ 上也可积,且
    \[\left|\int_{a}^{b} f(x) \mathrm{d}x\right| \leqslant \int_{a}^{b} |f(x)| \mathrm{d}x\]
\end{theorem}

\begin{theorem}{积分的平均值定理}
    设函数 $f,g$ 在 $[a, b]$ 上连续,$g$ 在 $[a, b]$ 上不改变符号,则 $\exists \xi \in [a, b]$,使得
    \[\int_{a}^{b} f(x)g(x) \mathrm{d}x = f(\xi)\int_{a}^{b} g(x) \mathrm{d}x\]
\end{theorem}

\begin{corollary}
    设函数 $f$ 在 $[a, b]$ 上连续,则存在一个点 $\xi \in [a, b]$,使得
    \[\int_{a}^{b} f(x) \mathrm{d}x = f(\xi)(b - a)\]
\end{corollary}

%——————————————————————————————————%

\section{微积分基本定理}

\begin{theorem}
    设函数 $f$ 在 $[a, b]$ 上可积,则上限变动的积分 $F(x) = \displaystyle \int_a^x f(t)\mathrm{d}t$ 在 $[a, b]$ 上连续。
\end{theorem}

\begin{theorem}
    设函数 $f$ 在 $[a, b]$ 上可积,在一点 $x_0 \in [a, b]$ 处连续,则 $F$ 在 $x_0$ 处可导,且
    \[F'(x_0) = f(x_0)\]
\end{theorem}

\begin{theorem}{微积分基本定理}
    设函数 $f$ 在 $[a, b]$ 上连续,则
    \[\frac{\mathrm{d}}{\mathrm{d}x}\int_{a}^{x}f(t)\mathrm{d}t = f(x)\quad (a \leqslant x \leqslant b)\]
\end{theorem}

\begin{corollary}
    $[a, b]$ 上的连续函数一定有原函数。
\end{corollary}

\begin{theorem}
    设函数 $f$ 在 $[a, b]$ 上连续,$G$ 是 $f$ 在 $[a, b]$ 上的任一原函数,则
    \[\int_{a}^{b}f(x)\mathrm{d}x = G(b) - G(a)\]
\end{theorem}

\begin{theorem}{Newton-Leibniz 公式}
    若函数 $G$ 在 $[a, b]$ 上有连续的导函数,则
    \[\int_{a}^{x}G'(t)\mathrm{d}t = G(x) - G(a)\quad (a \leqslant x \leqslant b)\]
\end{theorem}

%——————————————————————————————————%

\section{分部积分与换元}

\begin{proposition}{分部积分法}
    对等式
    \[u\mathrm{d}v = \mathrm{d}(uv) - v\mathrm{d}u\]
    两边作积分,得
    \[\int_{a}^{b}u(x)v'(x)\mathrm{d}x = u(x)v(x)\left|_{a}^{b}\right. - \int_{a}^{b}v(x)u'(x)\mathrm{d}x\]
\end{proposition}

\begin{theorem}{Taylor 公式的积分余项}
    设函数 $f$ 在 $(a, b)$ 上有直到 $n + 1$ 阶的连续导函数,则对任意固定的 $x_0 \in (a, b)$,有
    \[f(x) = f(x_0) + \frac{f'(x_0)}{1!}(x - x_0) + \cdots + \frac{f^{(n)}(x_0)}{n!}(x - x_0)^n + R_n(x)\]
    其中
    \[R_n(x) = \frac{1}{n!}\int_{x_0}^{x}(x - t)^{n}f^{(n + 1)}(t)\mathrm{d}t\quad (a < x < b)\]
\end{theorem}

考虑积分
\[\int_{x_0}^{x}(x -t)^{n}f^{(n + 1)}(t)\mathrm{d}t\]
利用积分中值定理,可得 Taylor 公式的 Lagrange 余项
\[R_n(x) = \frac{f^{(n + 1)}(\xi)}{(n + 1)!}(x - x_0)^{n + 1}\]
以及 Cauchy 余项
\[R_n(x) = \frac{f^{(n + 1)}(\xi)}{n!}(x - \xi)^{n}(x - x_0)\]

\begin{theorem}{换元公式}
    设函数 $f$ 在区间 $I$ 上连续,$a, b \in I$,函数 $\varphi$ 在区间 $[\alpha,\beta]$ 上有连续的导函数,$\varphi([\alpha,\beta]) \subseteq I$,且 $\varphi(\alpha) = a,\varphi(\beta) = b$,则
    \[\int_{a}^{b}f(x)\mathrm{d}x = \int_{\alpha}^{\beta}f \circ\varphi(t)\varphi'(t)\mathrm{d}t\]
\end{theorem}



%——————————————————————————————————%

\section{可积性理论}

\begin{definition}
    设函数 $f$ 在 $[a, b]$ 上有界,上确界为 $M$,下确界为 $m$。令 $\omega = M - m$,称为 $f$ 在 $[a, b]$ 上的振幅。

    对 $[a, b]$ 的一个分割 $P$,在 $P$ 的第 $i$ 个子区间 $[x_{i - 1},x_i](i = 1,2, \ldots n)$ 上 $f$ 的上确界与下确界分部记为 $M_i$ 和 $m_i$,并令 $\omega_i = M_i - m_i$,称 $\omega_i$ 为 $f$ 在 $[x_{i - 1},x_i]$ 上的振幅。
\end{definition}

\begin{definition}
    $f$ 关于分割 $P$ 的上和
    \[\overline{S}(f,P) = \sum_{i = 1}^{n}M_{i}\Delta x_i\]
    以及下和
    \[\underline{S}(f,P) = \sum_{i = 1}^{n}m_{i}\Delta x_i\]
    取定分割 $P$ 后,积分和被上和与下和所界定,即
    \[\overline{S}(f,P) \leqslant  \sum_{i = 1}^{n}f(\xi_{i})\Delta x_i \leqslant \underline{S}(f,P)\]
    其中 $\xi_i \in [x_{i - 1},x_i](i = 1,2, \ldots ,n)$ 是任取的。
\end{definition}

\begin{theorem}
    设 $P$ 与 $P'$ 是 $[a, b]$ 的两个分割,其中 $P'$ 是在 $p$ 的分点上多加 $k$ 个分点而成。则
    \[\underline{S}(f,P) \leqslant \underline{S}(f,P') \leqslant \underline{S}(f,P) + k\omega|P|\]
    \[\overline{S}(f,P) \geqslant \overline{S}(f,P') \geqslant \overline{S}(f,P) - k\omega|P|\]
\end{theorem}

\begin{definition}{上积分与下积分}
    上和的下确界记为 $\overline{I}$,称为 $f$ 在 $[a, b]$ 上的上积分。下和的上确界记为 $\underline{I}$,称为 $f$ 在 $[a, b]$ 上的下积分。

    任何有界函数 $f$ 的上积分与下积分都存在,且
    \[\underline{S}(f,P_1) \leqslant \underline{I} \leqslant \overline{I} \leqslant \overline{S}(f,P_2)\]
    其中 $P_1$ 与 $P_2$ 是 $[a, b]$ 的任意分割。
\end{definition}

\begin{theorem}{Darboux 定理}
    对 $[a, b]$ 上的任意有界函数,有
    \begin{align*}
        \lim_{|P| \to 0}\overline{S}(f,P) = \overline{I} \\
        \lim_{|P| \to 0}\underline{S}(f,P) = \underline{I}
    \end{align*}
\end{theorem}

\begin{theorem}
    设函数 $f:[a, b] \to \mathbb{R}$ 有界,则以下三个条件等价:
    \begin{enumerate}
        \item  $f$ 在 $[a, b]$ 上可积;
        \item $\lim_{|P| \to 0}\sum_{i = 1}^{n}\omega_i \Delta_i = 0$,其中 $\omega_i = M_i - m_i$ 是 $f$ 在 $[x_{i - 1},x_i](i = 1,2, \ldots n)$ 上的振幅。
        \item $\underline{I} = \overline{I}$。
    \end{enumerate}
\end{theorem}

\begin{theorem}
    设 $f$ 是定义在 $[a, b]$ 上的单调函数,则 $f$ 在 $[a, b]$ 上可积。
\end{theorem}

\begin{theorem}
    设 $f:[a, b] \to \mathbb{R}$ 是连续函数,则 $f$ 在 $[a, b]$ 上可积。
\end{theorem}

%——————————————————————————————————%

\section{Lebesgue 定理}

\begin{definition}
    设集合 $A$ 由实数组成。若对任意给定的 $\varepsilon > 0$,存在至多可数个开区间 $\{I_n : n \in \mathbb{N}^*\}$ 组成 $A$ 的一个开覆盖,且 $\displaystyle \sum_{n = 1}^{\infty} |I_n| \leqslant \varepsilon$,则称 $A$ 为零测度集,简称零测集。其中 $|I_n|$ 表示开区间 $I_n$ 的长度。
\end{definition}

\hfill

\begin{example}
    若 $A$ 是至多可数集,则 $A$ 一定是零测集。
\end{example}

\hfill

\begin{example}
    任何长度不为 $0$ 的区间都不是零测集。
\end{example}

\hfill

\begin{proposition}{零测集的性质}
    \begin{enumerate}
        \item 至多可数个零测集的并集是零测集;
        \item 设 $A$ 是零测集。若 $B \subseteq A$,则 $B$ 也是零测集。
    \end{enumerate}
\end{proposition}

\begin{lemma}
    设 $\omega$ 是有界函数 $f$ 在 $[a, b]$ 上的振幅,则
    \[\omega = \sup\{|f(y_1) - f(y_2)|:y_1,y_2 \in [a, b]\}\]
\end{lemma}

\begin{lemma}
    函数 $f$ 在点 $x \in I$ 处连续的充分必要条件是 $\omega_f(x) = 0$。
\end{lemma}

\begin{definition}
    函数 $f$ 在 $[a, b]$ 上不连续点的全体记为 $D(f)$,即
    \[D(f) = \{x \in [a, b] : f\, \text{在}\, x\, \text{处不连续}\}\]
    对 $\delta > 0$,将 $\omega(x) \geqslant \delta$ 的集合记为 $D_{\delta}$,即
    \[D_{\delta} = \{x \in [a, b] : \omega_{f}(x) \geqslant \delta\}\]
\end{definition}

\begin{lemma}
    令 $\delta = \dfrac{1}{n}$,则
    \[D(f) = \bigcup_{n = 1}^{\infty}D_{1/n}\]
\end{lemma}

\begin{lemma}
    设函数 $f : [a, b] \to \mathbb{R}$。若存在一列区间 $(\alpha_j,\beta_j)(j = 1,2, \ldots ,n)$,使得 $D(f) \subseteq \cup_{j = 1}^{\infty}(\alpha_j,\beta_j)$,记 $K = [a, b]\\ \cup_{j = 1}^{\infty}(\alpha_j,\beta_j)$,则对 $\forall \varepsilon > 0$,一定 $\exists \delta > 0$,当 $x \in K, y\in [a, b]$ 且 $|x - y| < \delta$ 时,有
    \[|f(x) - f(y)| < \varepsilon\]
\end{lemma}

\begin{theorem}{Lebesgue 定理}
    设函数 $f$ 在 $[a, b]$ 上有界,则 $f$ 在 $[a, b]$ 上 Riemann 可积的充分必要条件是 $D(f)$ 为零测集。
\end{theorem}

\begin{corollary}
    若 $f$ 在 $[a, b]$ 上有界,且在 $[a, b]$ 上只有至多可数个简短点,则 $f$ 在 $[a, b]$ 上 Riemann 可积。
\end{corollary}

\begin{corollary}
    若 $f$ 在 $[a, b]$ 上可积,则 $|f|$ 在 $[a, b]$ 上也可积。
\end{corollary}

\begin{corollary}
    若 $f,g$ 在 $[a, b]$ 上可积,则 $fg$ 在 $[a, b]$ 上也可积。
\end{corollary}

\begin{corollary}
    若 $f$ 在 $[a, b]$ 上可积,$\dfrac{1}{f}$ 在 $[a, b]$ 上有定义且有界,则 $\dfrac{1}{f}$ 在 $[a, b]$ 上也可积。
\end{corollary}

\begin{corollary}
    若 $f$ 在 $[a, b]$ 上可积,则对任意 $[c,d] \subseteq [a, b]$, $f$ 在 $[c,d]$ 上也可积。
\end{corollary}

\begin{corollary}
    若 $c \in (a, b)$,则当 $f$ 在 $[a,c]$ 与 $[c, b]$ 上都可积时,$f$ 在 $[a, b]$ 上可积。
\end{corollary}

\begin{corollary}
    设 $f$ 在 $[a, b]$ 上可积。若 $g$ 在 $[a, b]$ 上除去有限个点 $x_1,x_2, \ldots x_n$ 外和 $f$ 相等,则 $g$ 在 $[a, b]$ 上也可积,且
    \[\int_{a}^{b}f(x)\mathrm{d}x = \int_{a}^{b}g(x)\mathrm{d}x\]
\end{corollary}

\begin{corollary}
    设 $f$ 在 $[a, b]$ 上连续,$\varphi$ 在 $[c,d]$ 上可积,且 $\varphi([c,d]) \subseteq [a, b]$,则 $f\circ\varphi$ 在 $[c,d]$ 上可积。
\end{corollary}

%——————————————————————————————————%

\section{反常积分}

Riemann 积分的推广称为反常积分或广义积分。反常积分分为两大类:积分无界的反常积分简称为无穷积分;无界函数的积分称为瑕积分。

\begin{definition}{无穷积分}
    \begin{enumerate}
        \item 设函数 $f$ 在 $[a, +\infty)$ 上有定义,对 $\forall b > a$,$f$ 在 $[a, b]$ 上可积。上限 $b$ 变动的积分
              \[\int_{a}^{b}f(x)\mathrm{d}x\]
              定义了 $[a, +\infty)$ 上的一个函数。若极限
              \[\lim_{b \to +\infty}\int_{a}^{b}f(x)\mathrm{d}x\]
              存在且有限,就把该极限记为
              \[\int_{a}^{+\infty}f(x)\mathrm{d}x\]
              并称上述积分收敛,称 $f$ 在 $[a,+\infty)$ 上可积。若极限不存在,也用该记号表示,但称积分发散。
        \item 设函数 $f$ 在 $(-\infty, b]$ 上有定义,对 $\forall b > a$,$f$ 在 $[a, b]$ 上可积。下限 $a$ 变动的积分
              \[\int_{a}^{b}f(x)\mathrm{d}x\]
              定义了 $(-\infty, b]$ 上的一个函数。若极限
              \[\lim_{a \to -\infty}\int_{a}^{b}f(x)\mathrm{d}x\]
              存在且有限,就把该极限记为
              \[\int_{-\infty}^{b}f(x)\mathrm{d}x\]
              并称上述积分收敛,称 $f$ 在 $(-\infty, b]$ 上可积。若极限不存在,也用该记号表示,但称积分发散。
        \item 若无穷积分 $\displaystyle \int_{-\infty}^{b}f(x)\mathrm{d}x$ 和 $\displaystyle \int_{a}^{+\infty}f(x)\mathrm{d}x$ 都收敛,则称无穷积分
              \[\int_{-\infty}^{+\infty}f(x)\mathrm{d}x\]
              收敛,并规定
              \[\int_{-\infty}^{+\infty}f(x)\mathrm{d}x = \int_{-\infty}^{b}f(x)\mathrm{d}x + \int_{a}^{+\infty}f(x)\mathrm{d}x\]
              称 $f$ 在 $(-\infty, +\infty)$ 上可积。
    \end{enumerate}
\end{definition}

\begin{theorem}{无穷积分的 Newton-Leibniz 公式}
    设函数 $f$ 在 $[a,+\infty)$ 上可积,且有原函数 $F$,则
    \[\int_{a}^{+\infty}f(x)\mathrm{d}x = F(+\infty) - F(a)\]
    若 $f$ 在 $(-\infty, b]$ 上可积,且有原函数 $F$,则
    \[\int_{-\infty}^{b}f(x)\mathrm{d}x = F(b) - F(-\infty)\]
    若 $f$ 在 $(-\infty, +\infty)$ 上可积,且有原函数 $F$,则
    \[\int_{-\infty}^{+\infty}f(x)\mathrm{d}x = F(+\infty) - F(-\infty)\]
    其中 $F(-\infty) = \lim_{x \to -\infty}F(x), F(+\infty) = \lim_{x \to +\infty}F(x)$。
\end{theorem}

\begin{definition}{瑕积分}
    设函数 $f$ 在 $(a,b]$ 上有定义,且
    \[\lim_{\varepsilon \to a^{+}}f(x) = \infty\]
    但对 $\varepsilon \in (0, b - a)$,$f$ 在 $[a + \varepsilon, b]$ 上可积。若极限
    \[\lim_{\varepsilon \to a^{+}}\int_{a + \varepsilon}^{b}f(x)\mathrm{d}x\]
    存在且有限,则称瑕积分
    \[\int_{a}^{b}f(x)\mathrm{d}x\]
    收敛,并定义
    \[\int_{a}^{b}f(x)\mathrm{d}x := \lim_{\varepsilon \to a^{+}}\int_{a + \varepsilon}^{b}f(x)\mathrm{d}x\]
    若极限不存在,则称该瑕积分发散,称 $a$ 为瑕点。
\end{definition}

%——————————————————————————————————%

\section{数值积分}

当可积函数的原函数过于复杂,或根本没有原函数时,使用数值计算的方式来求出积分的近似值,这就是数值积分。

数值积分常用方法:梯形法。

\begin{theorem}
    设 $f$ 在 $[a, b]$ 上有二阶连续导数,令
    \[M = \max_{a \leqslant x \leqslant b}|f''(x)|\]
    以及
    \[x_i = a + \frac{i}{n}(b - a)\quad (i = 0,1,2, \ldots ,n)\]
    则
    \[\left|\ \int_{a}^{b}f(x)\mathrm{d}x - \left(\frac{y_0 + y_n}{2} + \sum_{i = 1}^{n - 1}y_i\right)\frac{b - a}{n}\ \right| \leqslant \frac{(b - a)^3}{12n^2}M\]
\end{theorem}

%——————————————————————————————————%

\section{Riemann 积分的应用}

在几何学中的应用
\begin{enumerate}
    \item 计算平面图形的面积;
    \item 计算空间曲线的弧长;
    \item 计算空间区域的体积;
    \item 计算旋转曲面的面积。
\end{enumerate}

在物理学中的应用
\begin{enumerate}
    \item 计算引力;
    \item 计算做功。
\end{enumerate}

面积原理就是用积分来估计和式。

\begin{theorem}
    若 $x \geqslant m \in \mathbb{N}^*$ 时,$f$ 是一个非负的递增函数,则当 $\xi \geqslant m$ 时,有
    \[\left|\ \sum_{k = m}^{[\xi]}f(k) - \int_{m}^{\xi}f(x)\mathrm{d}x\ \right| \leqslant f(\xi)\]
\end{theorem}

\begin{theorem}
    若 $x \geqslant m \in \mathbb{N}^*$ 时,$f$ 是一个非负的递减函数,则极限
    \[\lim_{n \to \infty}\left(\sum_{k = m}^{n}f(k) - \int_{m}^{n}f(x)\mathrm{d}x\right) = \alpha\]
    存在,且 $0 \leqslant \alpha \leqslant f(m)$。进一步地,若 $\displaystyle \lim_{x \to +\infty}f(x) = 0$,则
    \[\left|\ \sum_{k = m}^{[\xi]}f(k) - \int_{m}^{\xi}f(x)\mathrm{d}x -\alpha\ \right| \leqslant f(\xi - 1)\]
    其中 $\xi \geqslant m + 1$。
\end{theorem}

\begin{theorem}{Young 不等式}
    设连续函数 $\varphi$ 在 $[0, +\infty)$ 上严格递增,且 $\varphi(0) = 0$,则必存在连续的反函数 $\varphi^{-1}$ 在 $[0, \varphi(+\infty))$ 上严格递增,且 $\varphi^{-1}(0) = 0$。对 $\forall a > 0, 0 < b < \varphi(+\infty)$,下面的不等式成立
    \[ab \leqslant \int_{0}^{a}\varphi(x)\mathrm{d}x + \int_{0}^{b}\varphi^{-1}(y)\mathrm{d}y\]
    等号成立当且仅当 $b = \varphi(a)$(即 $a = \varphi^{-1}(b)$)。
\end{theorem}

令函数 $\varphi(x) = x^{p - 1}$,它的反函数 $\varphi^{-1}(y) = y^{q - 1}$,其中 $q = \dfrac{p}{p - 1} > 1$。利用 Young 不等式,得
\[ab \leqslant \frac{1}{p}a^p + \frac{1}{q}b^q \quad (\frac{1}{p} +\frac{1}{q} = 1)\]
不等式中等号成立当且仅当 $a^p = b^q$。

设 $a = A^{1/p}, b = B^{1/q} (p, q > 0)$,得
\[A^{1/p}B^{1/q} \leqslant \frac{A}{p} + \frac{B}{q} \quad (\frac{1}{p} +\frac{1}{q} = 1)\]
不等式中等号成立当且仅当 $A = B$。

\begin{theorem}{Hölder 不等式}
    设 $a_1, a_2, \ldots ,a_n$ 和 $b_1, b_2, \ldots ,b_n$ 是两组不全为零的非负实数,则下面的不等式成立
    \[\sum_{i = 1}^{n}a_{i}b_{i} \leqslant \left(\sum_{i = 1}^{n}a_{i}^{p}\right)^{1/p}\left(\sum_{i = 1}^{n}b_{i}^{q}\right)^{1/q}\]
    其中 $p, q > 1$,且 $\frac{1}{p} + \frac{1}{q} = 1$。不等式中等号成立的充分必要条件是存在常数 $\lambda$,使得
    \[a_{i}^{p} = \lambda b_{i}^{q}\quad (i = 1,2,\cdots,n)\]
\end{theorem}

\begin{theorem}{Cauchy-Schwarz 不等式}
    当 Hölder 不等式中 $p = q = 2$ 时,
    \[\left(\sum_{i = 1}^{n}a_{i}b_{i}\right)^2 \leqslant \sum_{i = 1}^{n}a_{i}^{2}\sum_{i = 1}^{n}b_{i}^{2}\]
    等号成立的充分必要条件是 $a_i = \lambda b_i(i = 1,2,\cdots,n)$。
\end{theorem}

\begin{theorem}{Wallis 公式}
    \[\lim_{n \to \infty}\frac{1}{2n + 1}\left(\frac{2 \cdot 4 \cdots (2n)}{1 \cdot 3 \cdots (2n - 1)}\right)^2 = \frac{\pi}{2}\]
    或
    \[\sqrt{\pi} = \lim_{n \to \infty}\frac{(n!)^{2}2^{2n}}{(2n)!\sqrt{n}}\]
\end{theorem}

\begin{theorem}{Stirling 公式}
    对数列 $a_n = \dfrac{n!e^n}{n^{n + \frac{1}{2}}}(n = 1,2, \cdots )$ 应用 Wallis 公式,得
    \[n! \sim \sqrt{2n\pi}\left(\frac{n}{e}\right)^n \quad (n \to \infty)\]
    或一个更精确的形式
    \[n! = \left(\frac{n}{e}\right)^{n}\sqrt{2\pi n}e^{\frac{\theta_n}{4n}}\]
    其中 $\theta_n = 4n \ln\dfrac{n!}{\frac{n}{e}\sqrt{2\pi n}}(0 < \theta_n < 1)$。
\end{theorem}
%——————————————————————————————————%