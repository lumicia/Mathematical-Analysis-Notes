\chapter{函数的导数}

%——————————————————————————————————%

\section{导数和微分}
\begin{definition}{导数}
    设 $f$ 是定义在 $[a, b]$ 上的函数,$x\in (a, b)$。若 $x$ 从 $x_0$ 变化到 $x_0 + \Delta x$,存在 $a\in \mathbb{R}$ 使得
    \[\lim_{\Delta x\to 0}\frac{f(x_0+\Delta x) - f(x_0)}{\Delta x} = \lim_{\Delta x\to 0}\frac{\Delta f(x)}{\Delta x} = a\]
    则称 $f$ 在 $x_0$ 处可导(derivable),$a$ 称为 $f$ 在 $x_0$ 处的导数(derivative),记为 $f'(x_0) = a$。
\end{definition}

令 $x = x_0 + \Delta x$,则 $f'(x_0)$ 的等价定义

\[f'(x_0) := \lim_{x\to x_0}\frac{f(x) - f(x_0)}{x - x_0}\]

\begin{definition}
    若曲线 $y = f(x)$ 在 $x_0$ 处可导,则过点 $P(x_0, f(x_0))$ 的斜率为 $f'(x_0)$ 的直线
    \[y = f(x_0) + f'(x_0)(x - x_0)\]
    称为曲线 $y = f(x)$ 在点 $P$ 处的切线(tangent line)。

    过点 $P$ 且垂直于切线的直线称为曲线 $y = f(x)$ 在点 $P$ 处的法线(normal line)。

    切线和法线斜率的乘积等于 $-1$。
\end{definition}


\begin{definition}{左导数和右导数}
    设函数 $f$。
    \begin{enumerate}
        \item 若 $f$ 在 $x_0$ 的左邻域 $N_r^-(x_0)$ 有定义,且存在 $a\in \mathbb{R}$ 使得
              \[\lim_{\Delta x\to 0-}\frac{f(x_0+\Delta x) - f(x_0)}{\Delta x} = a\]
              则称函数 $f$ 在 $x_0$ 的左侧可导,$a$ 称为 $f(x)$ 在 $x_0$ 处的左导数(left derivative),记为 $f_{-}'(x_0)$。
        \item 若 $f$ 在 $x_0$ 的右邻域 $N_r^+(x_0)$ 有定义,且存在 $a\in \mathbb{R}$ 使得
              \[\lim_{\Delta x\to 0+}\frac{f(x_0+\Delta x) - f(x_0)}{\Delta x} = a\]
              则称函数 $f$ 在 $x_0$ 的右侧可导,$a$ 称为 $f(x)$ 在 $x_0$ 处的右导数(left derivative),记为 $f_{+}'(x_0)$。
    \end{enumerate}
    左导数和右导数统称为单侧导数(one-sided derivative)。
\end{definition}

\begin{proposition}
    设函数 $f$ 在邻域 $N_r(x_0)$ 有定义。则 $f(x)$ 在 $x_0$ 处可导且 $f'(x_0) = a$ 当且仅当
    \[f_{-}'(x_0) = f_{+}'(x_0) = a\]
\end{proposition}

\begin{theorem}{可导和连续的关系}
    设函数 $f$ 在 $x_0$ 处可导,则 $f(x)$ 在 $x_0$ 处连续。
\end{theorem}
\begin{remark}
    函数在某处连续是在该处可导的必要条件,但不是充分条件,即函数连续不一定可导。
\end{remark}

\hfill

\begin{example}
    函数 $f(x) = |x|$ 在 $x = 0$ 处不可导。
\end{example}

\hfill

设函数 $f$ 在 $x_0$ 处可导。由导数定义,和 Landau 记号,
\[\lim_{\Delta x\to 0}\frac{\Delta f(x_0)}{\Delta x} = f'(x_0) \iff \frac{\Delta f(x_0)}{\Delta x} - f'(x_0) = o(1),\, \Delta x\to 0\]
则得
\begin{equation}
    \Delta f(x_0) = f'(x_0)\Delta x + o(\Delta x),\, \Delta x\to 0
\end{equation}
或写成
\begin{equation}
    f(x_0 + \Delta x) = f(x_0) + f'(x_0)\Delta x + o(\Delta x),\, \Delta x\to 0
\end{equation}
这两个等式都可称为函数的\textbf{无穷小增量公式}。$f'(x_0)\Delta x$ 称为 $\Delta f(x)$ 的线性主部(linear principal part),它的系数恰是 $f(x)$ 在 $x_0$ 处的导数。

\begin{definition}
    设函数 $f$ 在 $x_0$ 处的一个邻域 $N(x_0)$ 上有定义。若存在 $\lambda\in \mathbb{R}$ 使得
    \[\Delta f(x_0) = \lambda\Delta x + o(\Delta x),\, \Delta x\to 0\]
    则称函数 $f$ 在 $x_0$ 处可微(differentiable)。关于 $\Delta x$ 的线性函数 $\lambda\Delta x$ 称为函数 $f$ 在 $x_0$ 处的微分(differential),记为
    \[\mathrm{d}f(x_0) = \lambda\Delta x\]
    其中 $\lambda = f'(x_0)$。特别地,对于正比例函数 $g(x) = x$,由于 $g'(x) = 1$,故
    \[\mathrm{d}g(x) = \mathrm{d}x = \Delta x\]
    于是 $\mathrm{d}f(x_0) = f'(x_0)\Delta x$。
\end{definition}

对一元函数来说,可导和可微等价,但多元函数并非如此。

由于 $\mathrm{d}f(x) = f'(x)\mathrm{d}x$,故导数可以记为
\begin{equation}
    f'(x) = \frac{\mathrm{d}f(x)}{\mathrm{d}x}
\end{equation}

左边称为 Lagrange 记号,右边称为 Leibniz 记号。

若将 $\displaystyle \frac{\mathrm{d}}{\mathrm{d}x}$ 视为算子(operator),则导数还可以记为
\[\frac{\mathrm{d}}{\mathrm{d}x}f(x) = D_{x}f(x)\]

右边的记号称为 Euler 记号。

函数微分的几何意义:函数 $f$ 在 $x_0$ 处的微分就是曲线 $y = f(x)$ 在 $x_0$ 处的“切线函数”的增量。

\begin{theorem}{微分最佳逼近定理}
    设函数 $f$ 在 $x_0$ 处可微,$l$ 是 $f$ 在 $x_0$ 处的切线,有
    \[l(x) = f(x_0) + f'(x_0)(x - x_0),\, x\to x_0\]

    则对于任一不等于 $l(x)$ 的一次多项式 $L(x)$ 都存在 $\delta > 0$ 使得当 $0 < |x - x_0| < \delta$ 时,有
    \[|f(x) - l(x)| < |f(x) - L(x)|\]
\end{theorem}
函数的微分是在局部逼近效果最佳的一次多项式。

\begin{theorem}{导数的四则运算}
    设函数 $f$ 和 $g$ 在 $x$ 处可导,则 $f \pm g$,$fg$ 在 $x$ 处也可导。若 $g \ne 0$,则函数 $\dfrac{f}{g}$ 在 $x$ 处也可导,
    \begin{enumerate}
        \item $(f \pm g)'(x) = f'(x) \pm g'(x)$
        \item $(fg)'(x) = f'(x)g(x) + f(x)g'(x)$
        \item $\left(\dfrac{f}{g}\right)'(x) = \dfrac{f'(x)g(x) - f(x)g'(x)}{g^2(x)}$
    \end{enumerate}
\end{theorem}

\begin{theorem}{复合函数的链式求导法则}
    设函数 $f$ 在 $x_0$ 处可导,函数 $g$ 在 $t_0 = f(x_0)$ 处可导,则函数 $g \circ f$ 在 $x_0$ 处可导,且
    \[(g \circ f)'(x_0) = g'(f(x_0)) \cdot f'(x_0)\]
\end{theorem}

\begin{theorem}
    设函数 $y = f(x)$ 在区间 $I$ 上连续且严格单调,若 $f(x)$ 在 $x_0\in I$ 处可导,且 $f'(x_0) \ne 0$,则 $f(x)$ 的反函数 $x = f^{-1}(y)$ 在 $y_0 = f(x_0)$ 处可导,且
    \[\left(f^{-1}\right)'(y_0) = \dfrac{1}{f'(x_0)}\]
\end{theorem}

\begin{table}
    \centering
    \caption{初等函数导函数表}
    \renewcommand\arraystretch{1.8}
    \begin{tabular}{|c|c|c|c|}
        \hline
        类型                        & 原函数                                              & 导函数                       & 备注                                           \\
        \hline
        常值函数                    & $C$                                                 & $0$                          & $C\in \mathbb{R}$                              \\
        \hline
        \multirow{3}{*}{幂函数}     & $x^m$                                               & $mx^{m - 1}$                 & $m \in \mathbb{N}$                             \\
                                    & $x^{-m}$                                            & $-mx^{-m - 1}$               & $m \in \mathbb{N},x \ne 0$                     \\
                                    & $x^\mu$                                             & $\mu x^{\mu - 1}$            & $\mu \in \mathbb{R}, x > 0$                    \\
        \hline
        \multirow{2}{*}{指数函数}   & $e^x$                                               & $e^x$                        &                                                \\
                                    & $a^x$                                               & $a^x\ln a$                   & $a > 0, a \ne 1$                               \\
        \hline
        \multirow{2}{*}{对数函数}   & $\ln |x|$                                           & $\dfrac{1}{x}$               & $x \ne 0$                                      \\[8pt]
                                    & $\log_a|x|$                                         & $\dfrac{1}{x\ln a}$          & $a > 0, a \ne 1, x \ne 0$                      \\[5pt]
        \hline
        \multirow{4}{*}{三角函数}   & $\sin x$                                            & $\cos x$                     &                                                \\
                                    & $\cos x$                                            & $-\sin x$                    &                                                \\
                                    & $\tan x$                                            & $\dfrac{1}{\cos^{2}x}$       & $x \ne k\pi + \dfrac{\pi}{2}, k\in \mathbb{Z}$ \\
                                    & $\cot x$                                            & $-\dfrac{1}{\sin^{2}x}$      & $x \ne k\pi, k\in \mathbb{Z}$                  \\[5pt]
        \hline
        \multirow{4}{*}{反三角函数} & $\arcsin x$                                         & $\dfrac{1}{\sqrt{1 - x^2}}$  & $|x| < 1$                                      \\
                                    & $\arccos x$                                         & $-\dfrac{1}{\sqrt{1 - x^2}}$ & $|x| < 1$                                      \\
                                    & $\arctan x$                                         & $\dfrac{1}{1 + x^2}$         &                                                \\
                                    & $\mathrm{arccot}\, x$                               & $-\dfrac{1}{1 + x^2}$        &                                                \\[5pt]
        \hline
        \multirow{3}{*}{双曲函数}   & $\sinh x$                                           & $\cosh x$                    &                                                \\
                                    & $\cosh x$                                           & $\sinh x$                    &                                                \\
                                    & $\tanh x$                                           & $\dfrac{1}{\cosh^{2}x}$      &                                                \\[5pt]
        \hline
        \multirow{3}{*}{反双曲函数} & $\ln (x + \sqrt{x^2 + 1})$                          & $\dfrac{1}{\sqrt{x^2 + 1}}$  &                                                \\
                                    & $\ln (x + \sqrt{x^2 - 1})$                          & $\dfrac{1}{\sqrt{x^2 - 1}}$  & $x \geqslant 1$                                \\
                                    & $\dfrac{1}{2}\ln \left(\dfrac{1 + x}{1 - x}\right)$ & $\dfrac{1}{1 - x^2}$         & $|x| < 1$                                      \\[5pt]
        \hline
    \end{tabular}
\end{table}

\begin{definition}{高阶导数}
    设函数 $f$ 在区间 $I$ 上可导,则 $f'(x)\,(\forall \in I)$ 称为 $f(x)$ 的导函数。若 $f'(x)$ 在 $I$ 上仍可导,则 $f'(x)$ 的导函数称为 $f(x)$ 的二阶导函数(second order derivative),记为 $f''(x)$。以此类推,定义 $f(x)$ 的 $n$ 阶导函数($n$-order derivatve),记为 $f^{(n)}(x)$。
\end{definition}

设函数 $y = f(x)$,$y$ 的二阶微分记为 $\mathrm{d}^2$,则
\[\mathrm{d}^2f(x) = \mathrm{d}[\mathrm{d}f(x)] = \mathrm{d}[f'(x)]\mathrm{d}x = [f'(x)\mathrm{d}x]'\mathrm{d}x = [f''(x)\mathrm{d}x + f'(x)(\mathrm{d}x)']\mathrm{d}x = f''(x)(\mathrm{d}x)^2\]

约定 $(\mathrm{d}x)^ = \mathrm{d}x^2$,则
\[f''(x) = \frac{\mathrm{d}^{2}f(x)}{\mathrm{d}x^2} = \frac{\mathrm{d}^{2}y}{\mathrm{d}x^2}\]

类似地,$n$ 阶导数的微分形式为
\[f^{(n)}(x) = \frac{\mathrm{d}^{n}y}{\mathrm{d}x^n}\]

\begin{definition}{光滑函数}
    设区间 $I$。所有在 $I$ 上连续的函数组成的集合记为 $C(I)$ 或 $C^0(I)$。所有在 $I$ 上 $n$ 次连续可导的函数组成的集合记为 $C^{(n)}(I)$。所有在 $I$ 上任意次可导的函数组成的集合记为 $C^\infty(I)$。这样的函数称为 $I$ 上的光滑函数(smooth function)。在不强调区间 $I$ 时可以记为 $C^{(n)}$。
\end{definition}

\begin{theorem}{Leibniz 公式}
    设函数 $f, g \in C^{(n)}(I)$,则 $fg \in C^{(n)}(I)$,且
    \[(fg)^{(n)} = \sum_{k = 0}^{n} \mathrm{C}_{n}^{k}f^{(n - k)}g^{(k)} \]

    规定 $f^{(0)} = f, g^{(0)} = g$。
\end{theorem}
%——————————————————————————————————%

\section{中值定理}

\begin{definition}{极大值和极小值}
    设函数 $f:(a,b)\to \mathbb{R}$。
    \begin{enumerate}
        \item 若对点 $x_0 \in (a,b)$ 存在 $\delta > 0$,使得 $\Delta = (x_0 - \delta,x_0 + \delta) \subseteq (a,b)$,并且当 $x \in \Delta$ 时,$f(x_0)\geqslant f(x)$,那么称 $f(x_0)$ 是 $f$ 在 $(a,b)$ 上的局部极大值(local maximum),$x_0$ 称为 $f$ 的极大值点(local maximum point)。
        \item 若对点 $x_0 \in (a,b)$ 存在 $\delta > 0$,使得 $\Delta = (x_0 - \delta,x_0 + \delta) \subseteq (a,b)$,并且当 $x \in \Delta$ 时,$f(x_0)\leqslant f(x)$,那么称 $f(x_0)$ 是 $f$ 在 $(a,b)$ 上的极小值(local minimum),$x_0$ 称为 $f$ 的极小值点(local minimum point)。
    \end{enumerate}
    局部极大值和局部极小值统称为局部极值(local extremum),局部极大值点和局部极小值点统称为局部极值点(local extremum point)。
\end{definition}

\begin{theorem}{Fermat 引理}
    若函数 $f:(a,b)\to \mathbb{R}$ 在其局部极值点 $x_0 \in (a,b)$ 处可导,则必有 $f'(x) = 0$。
\end{theorem}

\begin{definition}{驻点}
    满足 $x_0 \in (a,b)$ 且 $f'(x) = 0$ 的点 $x_0$ 称为函数 $f$ 的驻点(critical point)。
\end{definition}

\begin{theorem}{Rolle 中值定理}
    设函数 $f$ 在 $[a,b]$ 上连续,在 $(a,b)$ 上可导,且 $f(a) = f(b)$,则存在点 $\xi \in (a,b)$,使得 $f'(\xi) = 0$。
\end{theorem}

\begin{lemma}
    设函数 $f$ 在 $[a,b]$ 上连续,在 $(a,b)$ 上可导,且 $\lambda(a) = 1, \lambda(b) = 0$,则必存在点 $\xi \in (a,b)$,使得
    \[f'(\xi) = \lambda'(\xi)(f(a) - f(b))\]
\end{lemma}

\begin{theorem}{Lagrange 中值定理}
    设函数 $f$ 在 $[a,b]$ 上连续,在 $(a,b)$ 上可导,则存在点 $\xi \in (a,b)$,使得
    \[\frac{f(b) - f(a)}{b - a} = f'(\xi)\]
\end{theorem}

\begin{corollary}
    设函数 $f$ 在 $[a,b]$ 上连续,在 $(a,b)$ 上可导,则 $f$ 在 $[a,b]$ 上为常数的充分必要条件是 $f' = 0$ 在 $(a,b)$ 上成立。
\end{corollary}

\begin{theorem}{Cauchy 中值定理}
    设函数 $f$ 和 $g$ 在区间 $[a,b]$ 上连续,在区间 $(a,b)$ 上可导,且当 $x \in (a,b)$ 时,$g'(x) \ne 0$,此时必存在一点 $\xi \in (a,b)$,使得
    \[\frac{f(b) - f(a)}{g(b) - g(a)} = \frac{f'(\xi)}{g'(\xi)}\]
\end{theorem}

\begin{theorem}{Darbux 定理}
    若 $f$ 在 $[a,b]$ 上可导,则:
    \begin{enumerate}
        \item 导函数 $f'$ 可以取到 $f'(a)$ 与 $f'(b)$ 之间的一切值;
        \item $f'$ 没有第一类间断点。
    \end{enumerate}
\end{theorem}




%——————————————————————————————————%

\section{函数的单调性、极值和凸性}
\subsection{单调性}

\begin{theorem}
  设函数 $f$ 在 $[a,b]$ 上连续,在 $(a,b)$ 上可导,则 $f$ 在 $[a,b]$ 上递增(递减)的充分必要条件是 $f' \geqslant 0\, (\leqslant 0)$ 在区间 $(a,b)$ 上成立。
\end{theorem}

\begin{theorem}
  设函数 $f$ 在 $[a,b]$ 上连续,在 $(a,b)$ 上可导,若 $f' > 0\, (f' < 0)$ 在 $(a,b)$ 上成立,则 $f$ 在 $[a,b]$ 上是严格递增(严格递减)的。
\end{theorem}

\begin{theorem}
  设函数 $f$ 在 $[a,b]$ 上连续,在 $(a,b)$ 内除了有限个点以外,有正(负)的导数,则 $f$ 在 $[a,b]$ 上严格递增(严格递减)。
\end{theorem}

\begin{theorem}
  设函数 $f$ 在 $[a,b]$ 上连续,在 $(a,b)$ 上可导,则 $f$ 在 $[a,b]$ 上严格递增(严格递减)的充分必要条件是
    \begin{enumerate}
        \item 当 $x \in (a,b)$ 时,$f' \geqslant 0\, (\leqslant 0)$;
        \item 在 $(a,b)$ 的任意开子区间上,$f' \ne 0$。
    \end{enumerate}
\end{theorem}

\subsection{极值}

\begin{theorem}
  设函数 $f$ 在 $[a,b]$ 上连续,$x_0 \in (a,b)$。
  \begin{enumerate}
    \item 若 $\exists \delta > 0$,使得在 $(x_0 - \delta,x_0)$ 上 $f' > 0$,而在 $x_0,x_0 + \delta$ 上 $f' < 0$,则 $f(x_0)$ 是 $f$ 的一个严格极大值。即当 $0 < |x - x_0| < \delta$ 时,$f(x) < f(x_0)$。 
    \item 若 $\exists \delta > 0$,使得在 $(x_0 - \delta,x_0)$ 上 $f' < 0$,而在 $x_0,x_0 + \delta$ 上 $f' > 0$,则 $f(x_0)$ 是 $f$ 的一个严格极大值。即当 $0 < |x - x_0| < \delta$ 时,$f(x) > f(x_0)$。 
  \end{enumerate}
\end{theorem}

\begin{theorem}
  设函数 $f$ 在 $[a,b]$ 上连续,$x_0 \in (a,b)$ 是 $f$ 的一个驻点。设 $f''$ 存在,则
  \begin{enumerate}
    \item 当 $f''(x_0) < 0$ 时,$f(x_0)$ 是 $f$ 的一个严格极大值;
    \item 当 $f''(x_0) > 0$ 时,$f(x_0)$ 是 $f$ 的一个严格极小值;
  \end{enumerate}
\end{theorem}

\subsection{凸性}

\begin{definition}
  设函数 $f$ 在区间 $I$ 上有定义。若对 $\forall x_1,x_2 \in I, x_1 \ne x_2$,以及对 $\forall \lambda_1 \lambda_2 > 0$,且 $\lambda_1 + \lambda_2 = 1$ 都有
  \[f(\lambda_{1}x_1 + \lambda_{2}x_2) \leqslant \lambda_{1}f(x_1) + \lambda_{2}f(x_2)\]
  则称 $f$ 是 $I$ 上的凸函数(convec function)。

  如果上述不等式对任意 $x_1 \ne x_2$ 以及 $\lambda_1,\lambda_2 > 0(\lambda_1 + \lambda_2 = 1)$ 不等号总成立,则 $f$ 上 $I$ 上的严格凸函数。
\end{definition}

\begin{theorem}
  设函数 $f$ 是区间 $I$ 上的凸函数,则对 $\forall x_1,x_2, \ldots ,x_n \in I$ 及 $\lambda_1,\lambda_2, \ldots ,\lambda_n > 0$,且 $\lambda_1 + \lambda_2 + \cdots + \lambda_n = 1$ 都有
  \[f(\sum_{i = 1}^{n}\lambda_{i}x_i) \leqslant \sum_{i = 1}^{n}\lambda_{i}f(x_i)\]
    若 $f$ 是 $I$ 上的严格凸函数,则当 $x_1,x_2, \ldots ,x_n$ 不全相等时,有
  \[f(\sum_{i = 1}^{n}\lambda_{i}x_i) < \sum_{i = 1}^{n}\lambda_{i}f(x_i)\]
\end{theorem}

\begin{theorem}{Jensen 不等式}
  设函数 $f$ 是区间 $I$ 上的凸函数,则对 $\forall x_1,x_2, \ldots ,x_n \in I$ 及 $\beta_1,\beta_2, \ldots ,\beta_n >0$,有 
  \[f(\sum_{i = 1}^{n}\beta_{i}x_{i}/\sum_{i = 1}^{n}\beta_i) \leqslant \frac{\displaystyle \sum_{i = 1}^{n}\beta_{i}f(x_i)}{\displaystyle \sum_{i = 1}^{n}\beta_i}\]
\end{theorem}

\begin{theorem}
  设函数 $f$ 是区间 $I$ 上的凸函数,当且仅当对任意 $(x_1,x_2) \subseteq I$ 及对任意 $x \in (x_1,x_2)$,有
  \[\frac{f(x) - f(x_1)}{x - x_1} \leqslant \frac{f(x_2) - f(x_1)}{x_2 - x_1} \leqslant \frac{f(x_2) - f(x)}{x_2 - x}\]
  $f$ 是 $I$ 上的严格凸函数,当且仅当上式中出现的都是严格的不等号。
\end{theorem}

\begin{theorem}
  设函数 $f$ 在 $[a,b]$ 上连续,在 $(a,b)$ 上可导,则 $f$ 在 $[a,b]$ 上为凸函数(严格凸函数)的充分必要条件是 $f'$ 在 $(a,b)$ 上递增(严格递增)。
\end{theorem}

\begin{theorem}
  设函数 $f$ 在 $[a,b]$ 上连续,在 $(a,b)$ 上有二阶导数,则 $f$ 在 $[a,b]$ 上为凸函数的充分必要条件是 $f'' \geqslant 0$ 在 $(a,b)$ 上成立。

  $f$ 在 $[a,b]$ 上为严格凸函数的充分必要条件是 $f'' \geqslant 0$ 在 $(a,b)$ 上成立,且在 $(a,b)$ 的任意开子区间内 $f''$ 不恒等于 $0$。
\end{theorem}
%——————————————————————————————————%

\section{L'Hospital 法则}






%——————————————————————————————————%

\section{Taylor 定理}






%——————————————————————————————————%
