\chapter{极限}

%——————————————————————————————————%

\section{数列极限}

数列指一组由无限个数形成的排列,记为 $\{a_n\}$。

若数列在 $n$ 变得越来越大时,通项 $a_n$ 越来越接近某个常数 $a$,则称 $\{a_n\}$ 是收敛数列。

\begin{definition}{数列极限的定义}
    设数列 $\{a_n\}$,$a \in \mathbb{R}$。若对 $\forall \varepsilon > 0, \exists N \in \mathbb{N}^*$,使得当 $n > N$ 时,有
    \[|a_n - a| < \varepsilon\]
    则称数列 $\{a_n\}$ 在 $n$ 趋于无穷大时以 $a$ 为极限,记为
    \[\lim_{n \to \infty} a_n = a\]
    简记为 $a_n \to a(n \to \infty)$。也称数列 $\{a_n\}$ 收敛于 $a$。

    存在极限的数列称为收敛数列,否则称为发散数列。
\end{definition}

\hfill

\begin{example}
    对 $\forall \alpha > 0$,有
    \[\lim_{n \to \infty}\frac{1}{n^{\alpha}} = 0\]
\end{example}

\hfill

\begin{example}
    当 $|q| < 1$ 时,有
    \[\lim_{n \to \infty}q^n = 0\]
\end{example}

\hfill

\begin{example}
    \[\lim_{n \to \infty}n^{\frac{1}{n}} = 1\]
\end{example}

\hfill

\begin{definition}{邻域}
    对 $a \in \mathbb{R}, r > 0 \in \mathbb{R}$,以 $a$ 为中心,$r$ 为半径的邻域
    \[N_r(a) := \{ x \in \mathbb{R} : |x - a| < r\}\]
\end{definition}

\begin{definition}{用邻域定义数列极限}
    数列 $\{a_n\}$ 收敛于 $a$ 可以表示为:$\forall \varepsilon > 0,\exists N \in \mathbb{N}^*$,除有限项 $a_1,a_2, \ldots ,a_N$ 以外的所有项都位于邻域 $N_r(a)$ 中。
\end{definition}

\begin{theorem}
    数列的极限唯一。
\end{theorem}

\begin{definition}{有界数列}
    设数列 $\{a_n\}$。
    \begin{enumerate}
        \item 若 $\exists A \in \mathbb{R}$,使得 $a_n \leqslant A$ 对 $\forall n \in \mathbb{N}^*$ 成立,则称 $\{a_n\}$ 有上界,$A$ 是 $\{a_n\}$ 的一个上界。
        \item 若 $\exists a \in \mathbb{R}$,使得 $a_n \geqslant a$ 对 $\forall n \in \mathbb{N}^*$ 成立,则称 $\{a_n\}$ 有下界,$a$ 是 $\{a_n\}$ 的一个下界。
    \end{enumerate}
    既有上界又有下界的数列称为有界数列。
\end{definition}

\begin{theorem}
    收敛数列是有界数列。
\end{theorem}

\begin{definition}{数列的子列}
    设数列 $\{a_n\}, k_i \in \mathbb{N}^*(i = 1,2,3, \ldots)$,且满足 $k_1 < k_2 < k_3 < \cdots$,则数列 $\{a_{k_n}\}$ 称为 $\{a_n\}$ 的子列。
\end{definition}

\begin{theorem}
    设收敛数列 $\{a_n\}$ 的极限为 $a$,则 $\{a_n\}$ 的任一子列都收敛于 $a$。
\end{theorem}

\begin{corollary}
    数列 $\{a_n\}$ 收敛的充分必要条件是 $\{a_n\}$ 的偶数项子列 $\{a_{2n}\}$ 和奇数项子列 $\{a_{2n - 1}\}$ 都收敛,且极限相同。
\end{corollary}

\begin{theorem}{极限的四则运算}
    设收敛数列 $\{a_n\}$ 和 $\{b_n\}$。
    \begin{enumerate}[itemsep=0.5em]
        \item $\displaystyle \lim_{n \to \infty}(a_n \pm b_n) = \lim_{n \to \infty}a_n \pm \lim_{n \to \infty}b_n$;
        \item $\displaystyle \lim_{n \to \infty}a_{n}b_{n} = \lim_{n \to \infty}a_n \cdot \lim_{n \to \infty}b_n$;
        \item $\displaystyle \lim_{n \to \infty}ca_n = c\lim_{n \to \infty}a_n, c\in \mathbb{R}$;
        \item $\displaystyle \lim_{n \to \infty}\frac{a_n}{b_n} = \frac{\displaystyle \lim_{n \to \infty}a_n}{\displaystyle \lim_{n \to \infty}b_n}$,其中 $\displaystyle \lim_{n \to \infty}b_n \ne 0$;
        \item $\displaystyle \lim_{n \to \infty}|a_n| = |\lim_{n \to \infty}a_n|$。
    \end{enumerate}
\end{theorem}

\begin{definition}{无穷小数列}
    若数列 $\{a_n\}$ 的极限为 $0$,则称 $\{a_n\}$ 为无穷小数列,简称无穷小。
\end{definition}

\begin{theorem}{无穷小的性质}
    \begin{enumerate}
        \item $\{a_n\}$ 为无穷小的充分必要条件是 $\{|a_n|\}$ 为无穷小;
        \item 两个无穷小之和(之差)仍为无穷小;
        \item 若 $\{a_n\}$ 为无穷小,$\{c_n\}$ 为有界数列则 $\{c_{n}a_{n}\}$ 也为无穷小;
        \item 设 $0 \leqslant a_n \leqslant b_n(n \in \mathbb{N}^{*})$,若 $\{b_n\}$ 为无穷小,则 $\{a_n\}$ 也为无穷小。
        \item $\displaystyle \lim_{n \to \infty}a_n = a$ 的充分必要条件是 $\{a_n - a\}$ 为无穷小。
    \end{enumerate}
\end{theorem}

\hfill

\begin{example}
    若收敛数列 $\{a_n\}$ 的极限为 $a$,则
    \[\lim_{n \to \infty}\frac{a_1 + a_2 + \cdots + a_n}{n} = a\]
\end{example}

\hfill

\begin{theorem}{夹逼原理}
    设 $a_n \leqslant b_n \leqslant  c_n(n \in \mathbb{N}^{*})$,若 $\displaystyle \lim_{n \to \infty}a_n = \lim_{n \to \infty}c_n = a$,则
    \[\lim_{n \to \infty}b_n = a\]
\end{theorem}

\begin{theorem}
    \begin{enumerate}
        \item 设 $\displaystyle \lim_{n \to \infty}a_n = a, \alpha, \beta$ 满足 $\alpha < a < \beta$,则当 $n$ 充分大时,有 $a_n > \alpha$;同理当 $n$ 充分大时,有 $a_n < \beta$。
        \item 设 $\displaystyle \lim_{n \to \infty}a_n = a, \lim_{n \to \infty}b_n = b$,且 $a < b$,则当 $n$ 充分大时,一定有 $a_n < b_n$。
        \item 设 $\displaystyle \lim_{n \to \infty}a_n = a, \lim_{n \to \infty}b_n = b$,并且当 $n$ 充分大时有 $a_n \leqslant  b_n$,则 $a \leqslant b$。

    \end{enumerate}
\end{theorem}

\begin{definition}{数列极限的推广}
    设数列 $\{a_n\}$,
    \begin{enumerate}
        \item 若 $\{a_n\}$ 满足对 $\forall A > 0, \exists N \in \mathbb{N}^{*}$,使得当 $n > N$ 时,有 $a_n > A$,则称数列 $\{a_n\}$ 趋向于 $+\infty$(正无穷大),记为
              \[\lim_{n \to \infty}a_n = +\infty\]
        \item 若对 $\forall A > 0, \exists N \in \mathbb{N}^{*}$,使得当 $n > N$ 时,有 $a_n < -A$,则称数列 $\{a_n\}$ 趋向于 $-\infty$(负无穷大),记为
              \[\lim_{n \to \infty}a_n = -\infty\]
    \end{enumerate}
\end{definition}

\begin{definition}
    若 $\displaystyle \lim_{n \to \infty}|a_n| = +\infty$,则称数列 $\{a_n\}$ 趋向于 $\infty$,记为 $\displaystyle \lim_{n \to \infty}a_n = \infty$。
    极限为 $+\infty,-\infty$ 或 $\infty$ 的数列 $\{a_n\}$ 都统称为无穷大。
\end{definition}

\begin{theorem}{无穷大的性质}
    \begin{enumerate}
        \item 若 $\{a_n\}$ 是无穷大,则 $\{a_n\}$ 必然无界(注意逆命题不成立)。
        \item 从无界数列中一定能选出一个子列是无穷大。
        \item 若 $\displaystyle \lim_{n \to \infty}a_n = +\infty$(或 $-\infty,\infty$),则对 $\{a_n\}$ 的任意子列 $\{a_{k_n}\}$ 也有
              \[\lim_{n \to \infty}a_{k_n} = +\infty\ \text{(或}\ -\infty,\infty\text{)}\]
        \item 若 $\displaystyle \lim_{n \to \infty}a_n = +\infty,\lim_{n \to \infty}b_n = +\infty$,则
              \[\lim_{n \to \infty}(a_n + b_n) = +\infty \\
                  \lim_{n \to \infty}(a_{n}b_{n}) = +\infty\]
              对 $a_n - b_n$ 和 $\dfrac{a_n}{b_n}$ 不成立。
        \item $\{a_n\}$ 是无穷大的充分必要条件是 $\left\{\dfrac{1}{a_n}\right\}$ 是无穷小。
    \end{enumerate}
\end{theorem}

\begin{definition}{扩充的实数集}
    将 $+\infty, -\infty$ 放入实数集 $\mathbb{R}$ 中,得到扩充的实数集
    \[\mathbb{R}_{\infty} = \mathbb{R} \cup \{-\infty, +\infty\}\]
\end{definition}

\begin{definition}{单调数列}
    设数列 $\{a_n\}$。
    \begin{enumerate}
        \item 若 $\{a_n\}$ 满足 $a_n \leqslant a_{n + 1}\quad (n=1,2, \ldots)$,则称 $\{a_n\}$ 为递增数列;
        \item 若 $\{a_n\}$ 满足 $a_n < a_{n + 1}\quad (n=1,2, \ldots)$,则称 $\{a_n\}$ 为严格递增数列;
        \item 若 $\{a_n\}$ 满足 $a_n \geqslant a_{n + 1}\quad (n=1,2, \ldots)$,则称 $\{a_n\}$ 为递减数列;
        \item 若 $\{a_n\}$ 满足 $a_n > a_{n + 1}\quad (n=1,2, \ldots)$,则称 $\{a_n\}$ 为严格递减数列;
    \end{enumerate}
    (严格)递增或递减的数列统称为(严格)单调数列。
\end{definition}

\begin{theorem}{单调有界定理}
    单调有界的数列一定有极限。
\end{theorem}

\begin{theorem}{闭区间套定理}
    设一列闭区间 $I_n = [a_n, b_n](n = 1,2, \ldots)$。若 $I_n \supseteq I_{n + 1}(n = 1,2, \ldots)$,且 $\displaystyle \lim_{n \to \infty}(b_n - a_n) = 0$,则存在唯一的 $c \in [a_n, b_n](n = 1,2, \ldots)$,其中
    \[c = \lim_{n \to \infty}a_n = \lim_{n \to \infty}b_n\]
\end{theorem}

\begin{theorem}{自然对数的底 e}
    \[\mathrm{e} = \lim_{n \to \infty}\left(1 + \frac{1}{n}\right)^{n} = \lim_{n \to \infty}\left(1 + \frac{1}{1!} + \frac{1}{2!} + \cdots + \frac{1}{n!}\right)\quad (n \in \mathbb{N}^{*})\]
\end{theorem}

\begin{theorem}
    自然对数的底 $\mathrm{e}$ 是无理数。
\end{theorem}

\hfill

\begin{example}
    \[\lim_{n \to \infty}\left(1 - \frac{1}{n}\right)^{n} = \frac{1}{e}\]
\end{example}

\hfill

\begin{example}
    \[\lim_{n \to \infty}\left(1 + \frac{2}{n}\right)^{n} = e^2\]
\end{example}

\hfill

\begin{example}
    Euler 常数 $\gamma$:
    \[\gamma = \lim_{n \to \infty}\left( 1 + \frac{1}{2} + \cdots + \frac{1}{n} - \ln(n + 1) \right)\quad (n \in \mathbb{N}^{*})\]
\end{example}

\begin{definition}{基本列}
    设实数列 $\{a_n\}$。对 $\forall \varepsilon > 0$,若 $\exists N \in \mathbb{N}^{*}$,使得当 $m, n \in \mathbb{N}^{*}$ 且 $m, n > N$ 时,有
    \[|a_m - a_n| < \varepsilon\]
    则称数列 $\{a_n\}$ 是基本列或 Cauchy 列。

    若 $m > n$,令 $m = n + p$,则基本列的等价定义:对 $\forall \varepsilon > 0$,若 $\exists N \in \mathbb{N}^{*}$,使得当 $n > N$ 时,有
    \[|a_{n + p} - a_{n}| < \varepsilon\]
    对 $\forall p \in \mathbb{N}^{*}$ 成立,则称 $\{a_n\}$ 是基本列。
\end{definition}

\begin{lemma}
    从任一数列中必定可以选出一个单调子列。
\end{lemma}

\begin{theorem}{Bolzano-Weierstrass 定理}
    从任何有界数列中必定可选出一个收敛的子列。
\end{theorem}

\begin{remark}
    Bolzano-Weierstrass 定理又称为列紧性定理、致密性定理、聚点定理等。
\end{remark}

\begin{theorem}{Cauchy 收敛原理}
    数列 $\{a_n\}$ 收敛的充分必要条件是 $\{a_n\}$ 是基本列。
\end{theorem}

\begin{definition}{上界和下界}
    设由实数组成的集合 $E$。
    \begin{enumerate}
        \item 若 $\exists B \in \mathbb{R}$,使得对 $\exists x \in E$,有 $x \leqslant B$,称 $B$ 是 $E$ 的一个上界。
        \item 若 $\exists A \in \mathbb{R}$,使得对 $\exists x \in E$,有 $x \geqslant A$,称 $A$ 是 $E$ 的一个下界。
    \end{enumerate}
    既有上界又有下界的集合称为有界集。
\end{definition}

\begin{definition}{上确界和下确界}
    设非空有上界的集合 $E$。若 $\beta \in \mathbb{R}$ 满足
    \begin{enumerate}
        \item 对 $\forall x \in E$,有 $x \leqslant \beta$;
        \item 对 $\forall \varepsilon > 0$,必定可找到一个 $x_{\varepsilon} \in E$,使得 $x_{\varepsilon} > \beta - \varepsilon$。
    \end{enumerate}
    则称 $\beta$ 为集合 $E$ 的上确界,记为 $\beta = \sup E$。$\beta$ 是 $E$ 的最小上界。

    设非空有下界的集合 $E$。若 $\alpha \in \mathbb{R}$ 满足
    \begin{enumerate}
        \item 对 $\forall x \in E$,有 $x \geqslant \alpha$;
        \item 对 $\forall \varepsilon > 0$,必定可找到一个 $y_{\varepsilon} \in E$,使得 $y_{\varepsilon} < \alpha + \varepsilon$。
    \end{enumerate}
    则称 $\alpha$ 为集合 $E$ 的下确界,记为 $\alpha = \inf E$。$\alpha$ 是 $E$ 的最大下界。

\end{definition}

\begin{theorem}{确界原理}
    非空的有上界(下界)的集合必有上确界(下确界)。
\end{theorem}

\begin{definition}{开覆盖}
    设集合 $E \in \mathbb{R}$,开区间族 $\{I_{\lambda} : \lambda \in \Lambda\}$,其中 $\Lambda$ 是一个指标集。若
    \[E \subseteq \bigcup_{\lambda \in \Lambda} I_{\lambda}\]
    则称 $\{I_{\lambda} : \lambda \in \Lambda\}$ 是 $E$ 的一个开覆盖,记为 $C_E$。

    若 $E$ 的一个开覆盖 $C_{E}^{'} \subseteq C_{E}$,则称 $C_{E}^{'}$ 是 $C_E$ 的子覆盖。若 $C_{E}^{'}$ 只含有有限个开区间,则称 $C_{E}^{'}$ 是有限子覆盖。
\end{definition}

\begin{theorem}{Heine-Borel 定理}
    设有限闭区间 $[a,b]$,则 $[a,b]$ 的任一开覆盖 $\{I_{\lambda} : \lambda \in \Lambda\}$ 都存在一个有限子开覆盖。
\end{theorem}

\begin{definition}{极限点}
    数列 $\{a_n\}$ 的收敛子列 $\{a_{k_n}\}$ 的极限称为 $\{a_n\}$ 的极限点。
\end{definition}

\begin{remark}
    收敛数列只有一个极限点;有界发散数列有若干乃至无穷个极限点;无界发散数列除有限个极限点以外,还能以 $+\infty$ 或 $-\infty$ 为极限点。
\end{remark}

\begin{definition}{上极限和下极限}
    设数列 $\{a_n\}$,由 $\{a_n\}$ 全体极限点组成的集合 $E$。$\{a_n\}$ 的上极限和下极限分别记为
    \begin{align*}
        \sup E & = \limsup_{n \to \infty} a_n \\
        \inf E & = \liminf_{n \to \infty} a_n
    \end{align*}
    $\sup E$ 也可以记为 $a^{*}$ 或 $\displaystyle \varlimsup_{n \to \infty} a_n$,$\inf E$ 也可以记为 $a_{*}$ 或 $\displaystyle \varliminf_{n \to \infty} a_n$
\end{definition}

\begin{theorem}
    设数列 $\{a_n\}$,由 $\{a_n\}$ 全体极限点组成的集合 $E$。则对于上极限 $\sup E$,有
    \begin{enumerate}
        \item $\sup E \in E$;
        \item 若 $x > \sup E$,则 $\exists N \in \mathbb{N}^{*}$,使得当 $n \geqslant N$ 时,有 $a_n < x$;
        \item $\sup E$ 是唯一满足前两条性质的数。
    \end{enumerate}
    对于下极限 $\inf E$,有
    \begin{enumerate}
        \item $\inf E \in E$;
        \item 若 $x < \inf E$,则 $\exists N \in \mathbb{N}^{*}$,使得当 $n \geqslant N$ 时,有 $a_n > x$;
        \item $\inf E$ 是唯一满足前两条性质的数。
    \end{enumerate}
\end{theorem}

\begin{theorem}
    设数列 $\{a_n\}, \{b_n\}$。
    \begin{enumerate}
        \item $\displaystyle \liminf_{n \to \infty} a_n \leqslant \limsup_{n \to \infty} a_n$;
        \item $\displaystyle \lim_{n \to \infty} a_n = a$ 当且仅当 $\displaystyle \liminf_{n \to \infty} a_n = \limsup_{n \to \infty} a_n = a$;
        \item 若 $N \in \mathbb{N}^{*}$,当 $n > N$ 时有 $a_n \leqslant b_n$,则
              \begin{align*}
                  \liminf_{n \to \infty} a_n & \leqslant \liminf_{n \to \infty} b_n \\
                  \limsup_{n \to \infty} a_n & \leqslant \limsup_{n \to \infty} b_n
              \end{align*}
    \end{enumerate}
\end{theorem}

\begin{theorem}
    设数列 $\{a_n\}$。则
    \begin{enumerate}[itemsep=0.5em]
        \item $\displaystyle \left\{\inf_{k \geqslant n} a_k\right\}$ 是递增数列,$\displaystyle \left\{\sup_{k \geqslant n} a_k\right\}$ 是递减数列。
        \item $\displaystyle \lim_{n \to \infty}\inf_{k \geqslant n} a_k = \inf E$,$\displaystyle \lim_{n \to \infty}\sup_{k \geqslant n} a_k = \sup E$。
    \end{enumerate}
\end{theorem}

\hfill

\begin{example}
    \begin{gather*}
        \liminf_{n \to \infty} a_n + \liminf_{n \to \infty} b_n \leqslant \liminf_{n \to \infty} (a_n + b_n) \leqslant \liminf_{n \to \infty} a_n + \limsup_{n \to \infty} b_n \\
        \liminf_{n \to \infty} a_n + \limsup_{n \to \infty} b_n \leqslant \limsup_{n \to \infty} (a_n + b_n) \leqslant \limsup_{n \to \infty} a_n + \limsup_{n \to \infty} b_n
    \end{gather*}
\end{example}

\hfill

\begin{definition}{Toeplitz 矩阵}
    设无限矩阵 $T = (t_{nk})(n, k = 1,2, \ldots)$,若 $T$ 满足
    \begin{enumerate}
        \item $\exists M \in \mathbb{R}, \displaystyle \sum_{k = 1}^{\infty} |t_{nk}| \leqslant M$;
        \item $\displaystyle \lim_{n \to \infty}\sum_{k = 1}^{\infty} t_{nk} = 1$;
        \item $\forall k \in \mathbb{N}^{*}$,有 $\displaystyle \lim_{n \to \infty} t_{nk} = 0$。
    \end{enumerate}
    则称 $T$ 为 Toeplitz 矩阵。设数列 $\{a_n\}$。令
    \[x_n = \sum_{k = 1}^{n} t_{nk}a_{k}\]
    则 $x_n$ 称为关于 $T$ 的 Toeplitz 变换。
\end{definition}

\begin{theorem}{Toeplitz 定理}
    设 $n, k \in \mathbb{N}^{*}$ 时有 $t_{nk} \geqslant 0$,且 $\displaystyle \sum_{k = 1}^{n} t_{nk} = 1, \lim_{n \to \infty} t_{nk} = 0$。若 $\displaystyle \lim_{n \to \infty} a_n = a$,令
    \[x_n = \sum_{k = 1}^{n} t_{nk}a_{k}\]
    则
    \[\lim_{n \to \infty} x_n = a\]
\end{theorem}

\begin{theorem}{Stolz 定理}
    设 $\{b_n\}$ 是严格递增且趋于 $+\infty$ 的数列。若
    \[\lim_{n \to \infty}\frac{a_{n} - a_{n - 1}}{b_{n} - b_{n - 1}} = A\]
    则
    \[\lim_{n \to \infty}\frac{a_n}{b_n} = A\]
    其中 $A$ 可以是 $+\infty$ 或 $-\infty$。
\end{theorem}

\begin{remark}
    Stolz 定理的逆命题不成立。
\end{remark}

%——————————————————————————————————%

\section{函数}

\begin{definition}
    设集合 $A, B$。若 $f$ 是一种规律,使得对 $A$ 中的每个元素 $x$,$B$ 中都有唯一确定的元素 $f(x)$ 与之对应,则称 $f$ 是从 $A$ 到 $B$ 的映射,表示为
    \[f : A \to B\]
    集合 $A$ 称为映射 $f$ 的定义域;$f(x) \in B$ 称为 $x$ 在映射 $f$ 下的像或 $f$ 在 $x$ 上的值。
\end{definition}

\begin{definition}
    设映射 $f : A \to B$,且 $g : A \to B$。若对 $\forall x \in A$,均有 $f(x) = g(x)$,则称 $f$ 和 $g$ 相等,记为 $f = g$。
\end{definition}

\begin{definition}
    设映射 $f : A \to B$。若 $f(A) = B$,则称 $f$ 是从 $A$ 到 $B$ 上的满射,即 $B$ 中的任何元素都是 $A$ 中某一元素在 $f$ 下的像。
\end{definition}

\begin{definition}
    设映射 $f : A \to B$。若当 $x, y \in A$ 且 $x \ne y$ 时,有 $f(x) \ne f(y)$,则称 $f$ 为单射。
\end{definition}

\begin{remark}
    单射的等价定义:若 $f(x) = f(y)$,则 $x = y$。
\end{remark}

\begin{definition}
    若映射 $f : A \to B$ 既是满射又是单射,则称 $f$ 是双射或一一对应。
\end{definition}

\begin{definition}
    设双射 $f : A \to B$,定义 $f$ 的逆映射 $f^{-1} : B \to A$。
\end{definition}

\begin{definition}
    设映射 $f : A \to B$,$F \subseteq B$,则 $A$ 的子集
    \[f^{-1}(F) = \{ x \in A: f(x) \in F \}\]
    称为 $F$ 的原像或逆像。
\end{definition}

\begin{definition}
    设映射 $f : B \to C$,映射 $g$ 的定义域为 $A$。当 $x \in A = g^{-1}(B)$ 时,定义映射
    \[f \circ g(x) = f(g(x))\]
    则 $f \circ g: A \to C$ 称为映射 $f$ 与 $g$ 的复合。
\end{definition}

\begin{proposition}
    设映射 $f, g, h$ 都是从 $A$ 到 $A$ 的映射,$x \in A$,则
    \[f \circ (g \circ h)(x) = f(g \circ h(x)) = f(g(h(x))) = f \circ g(h(x)) = (f \circ g) \circ h(x)\]
    因此 $f \circ (g \circ h)$ 可以简写为 $f \circ g \circ h$。
\end{proposition}

\begin{definition}
    设映射 $f: A \to A$,则 $f$ 的 $n$ 次复合 $f \circ f \circ \cdots \circ f$ 可以简记为 $f^{n}$。
\end{definition}

\begin{definition}
    若集合 $A$ 到自身的映射 $I_A(x) = x$ 对 $\forall x \in A$ 成立,则称 $I_A$ 为 $A$ 上的恒等映射。一个集合的 $I_A$ 是唯一存在的。
\end{definition}

\begin{definition}
    设集合 $A, B$。若存在从 $A$ 到 $B$ 的一一对应,则称集合 $A$ 和 $B$ 有相同的势或基数。此时称 $A$ 与 $B$ 等价,记为 $A \simeq B$。
\end{definition}

\begin{proposition}
    等价关系的性质
    \begin{enumerate}
        \item $A \simeq A$;
        \item 若 $A \simeq B$,则 $B \simeq A$;
        \item 若 $A \simeq B$,且 $B \simeq C$,则 $A \simeq C$。
    \end{enumerate}
\end{proposition}

\begin{definition}
    设全体正整数的集合 $\mathbb{N}^{*}$。令
    \[N_n = \{ 1,2, \ldots ,n \}\]
    \begin{enumerate}
        \item 若 $\exists n \in \mathbb{N}^{*}$,使得集合 $A \simeq N_n$,则 $A$ 称为有限集。$\varnothing$ 也称为有限集。
        \item 若集合 $A$ 不是有限集,则称 $A$ 为无限集。
        \item 若 $A \simeq \mathbb{N}^{*}$,,则称 $A$ 为可数集。
        \item 若 $A$ 既不是有限集,也不是可数集,则称 $A$ 为不可数集。
        \item 若 $A$ 是有限集或可数集,则称 $A$ 为至多可数集。
    \end{enumerate}
\end{definition}

\begin{theorem}
    可数集 $A$ 的每一个无限子集都是可数集。
\end{theorem}

\begin{theorem}
    设至多可数集族 $\{ E_n (n = 1,2,3, \ldots) \}$,令
    \[S = \bigcup_{n = 1}^{\infty} E_n \]
    则 $S$ 是至多可数集。
\end{theorem}

\begin{theorem}
    $\mathbb{R}$ 中的全体有理数是可数的。
\end{theorem}

\begin{theorem}
    $[0, 1]$ 上的全体实数是不可数的。
\end{theorem}

\begin{proposition}
    平面直角坐标系中横纵坐标均为有理数的点 $(x, y)$ 称为有理点。则平面上的全体有理点组成的集合是可数集。
\end{proposition}

\begin{proposition}
    若复数 $x$ 满足多项式方程
    \[a_{0}x^{n} + a_{1}x^{n - 1} + \cdots + a_{n - 1}x + a_{n} = 0\]
    其中 $a_0 \neq 0, a_1, \ldots , a_n$ 都是整数,则 $x$ 称为代数数。则代数数全体是可数集。
\end{proposition}

\begin{definition}
    设映射 $f: X \to Y$。若集合 $X$ 和 $Y$ 都是由实数组成的集合,则 $f$ 称为函数。
\end{definition}

\begin{definition}
    设函数 $f$ 和 $g$ 的定义域分别为 $A$ 和 $B$。则在集合 $A \cup B$ 上可以定义 $f$ 和 $g$ 的四则运算。令 $x \in A \cup B$。
    \begin{enumerate}[itemsep=0.5em]
        \item $(f \pm g)(x) = f(x) \pm g(x)$;
        \item $(fg)(x) = f(x)g(x)$;
        \item $\left( \dfrac{f}{g} \right)(x) = \dfrac{f(x)}{g(x)}$。
    \end{enumerate}
\end{definition}

\begin{definition}
    若函数 $f: X \to Y$ 是一一对应,则 $f$ 的逆映射 $f^{-1}: Y \to X$,称 $f^{-1}$ 为 $f$ 的反函数。若
    \[y = f(x)\quad (x \in X)\]
    则
    \[x = f^{-1}(y)\quad (y \in Y)\]
    从而
    \begin{gather*}
        f \circ f^{-1}(y) = f(x) = y\quad (y \in Y) \\
        f^{-1} \circ f(x) = f^{-1}(y) = x\quad (x \in X)
    \end{gather*}
    因此 $f \circ f^{-1}$ 是 $Y$ 上的恒等映射,$f^{-1} \circ f$ 是 $X$ 上的恒等映射。
\end{definition}

\begin{definition}
    函数 $f: X \to Y$ 称为 $X$ 上的递增(递减)函数,若对 $\forall x_1, x_2 \in X$,只要 $x_1 < x_2$,便有 $f(x_1) \leqslant f(x_2)(f(x_1) \geqslant f(x_2))$。

    函数 $f: X \to Y$ 称为 $X$ 上的严格递增(递减)函数,若对 $\forall x_1, x_2 \in X$,只要 $x_1 < x_2$,便有 $f(x_1) < f(x_2)(f(x_1) > f(x_2))$。

    在 $X$ 上的(严格)递增或递减的函数统称为(严格)单调函数。
\end{definition}

\begin{theorem}
    设函数 $f$ 在定义域 $X$ 上是严格递增(递减)的,则反函数 $f^{-1}$ 必定存在,$f^{-1}$ 的定义域为 $f(X)$,且 $f^{-1}$ 在定义域上也是严格递增(递减)的。
\end{theorem}


%——————————————————————————————————%

\section{函数极限}

\begin{definition}
    若 $a \in \mathbb{R}, r \in \mathbb{R}^{+}$,则以 $a$ 为中心,$r$ 为半径的去心邻域
    \[\check{N}_{r}(a) = N_{r}(a)\setminus \{ a \}\]
\end{definition}

\begin{definition}
    设函数 $f$ 在 $x_0$ 的去心邻域 $\check{N}_{r}(x_0)$ 上有定义。设实数 $a$。若对 $\forall \varepsilon > 0$,$\exists \delta > 0$,使得对 $x \in \check{N}_{\delta}(x_0)$,均有
    \[|f(x) - a| < \varepsilon\]
    则称当 $x$ 趋于点 $x_0$ 时函数 $f(x)$ 有极限 $a$。
\end{definition}

\begin{definition}
    设 $a \in \mathbb{R}$。则以 $a$ 为中心,$r$ 为半径的左邻域
    \[N_{r}^{-}(a) = (a - r, a)\]
    右邻域
    \[N_{r}^{+}(a) = (a, a + r)\]
\end{definition}

\begin{definition}
    设实数 $a$。
    \begin{enumerate}
        \item 设函数 $f$ 在 $x_0$ 的左邻域 $N_{r}^{-}(x_0)$ 上有定义。若对 $\forall \varepsilon > 0$,$\exists \delta > 0$,使得对 $x \in \check{N}_{\delta}^{-}(x_0)$,均有
              \[|f(x) - a| < \varepsilon\]
              则称 $a$ 为 $f$ 在 $x_0$ 处的左极限,表示为
              \[a = \lim_{x \to x_{0}^{-}}f(x)\]
        \item 设函数 $f$ 在 $x_0$ 的右邻域 $N_{r}^{+}(x_0)$ 上有定义。若对 $\forall \varepsilon > 0$,$\exists \delta > 0$,使得对 $x \in \check{N}_{\delta}^{+}(x_0)$,均有
              \[|f(x) - a| < \varepsilon\]
              则称 $a$ 为 $f$ 在 $x_0$ 处的右极限,表示为
              \[a = \lim_{x \to x_{0}^{+}}f(x)\]
    \end{enumerate}
    左极限和右极限统称为单侧极限。
\end{definition}

\begin{theorem}
    设函数 $f$ 在 $\check{N}(x_0)$ 有定义。则 $\displaystyle \lim_{x \to x_0}f(x)$ 存在当且仅当
    \[\lim_{x \to x_{0}^{-}}f(x) = \lim_{x \to x_{0}^{+}}f(x)\]
    且 $\displaystyle \lim_{x \to x_0}f(x) = \lim_{x \to x_{0}^{-}}f(x) = \lim_{x \to x_{0}^{+}}f(x)$。
\end{theorem}

\begin{theorem}{Heine 定理}
    函数 $f$ 在 $x_0$ 处有极限 $a$ 的充分必要条件是,对任意收敛于 $x_0$ 的数列 $\{ x_n \neq x_0: n = 1,2,3, \ldots \}$,数列 $\{ f(x_n) \}$ 有极限 $a$。
\end{theorem}

\begin{theorem}
    若极限 $\displaystyle \lim_{x \to x_0}$ 存在,则它是唯一的。
\end{theorem}

\begin{theorem}
    若函数 $f$ 在 $x_0$ 处有极限,则 $\exists M > 0, \delta > 0$,使得当 $x \in \check{N}_{\delta}(x_0)$ 时,有
    \[|f(x)| < M\]
\end{theorem}

\begin{theorem}
    设 $\displaystyle \lim_{x \to x_0}f(x)$ 与 $\displaystyle \lim_{x \to x_0}g(x)$ 存在。则
    \begin{enumerate}[itemsep=0.5em]
        \item $\displaystyle \lim_{x \to x_0}(f \pm g)(x) = \lim_{x \to x_0}f(x) \pm \lim_{x \to x_0}g(x)$;
        \item $\displaystyle \lim_{x \to x_0}fg(x) = \lim_{x \to x_0}f(x) \cdot \lim_{x \to x_0}g(x)$;
        \item $\displaystyle \lim_{x \to x_0}\frac{f}{g}(x) = \frac{\displaystyle \lim_{x \to x_0}f(x)}{\displaystyle \lim_{x \to x_0}g(x)}$,其中 $\displaystyle \lim_{x \to x_0}g(x) \neq 0$。
    \end{enumerate}
\end{theorem}

\begin{theorem}{夹逼原理}
    设函数 $f, g, h$ 在 $\check{N}(x_0)$ 有定义,且满足
    \[f(x) \leqslant h(x) \leqslant g(x)\]
    若 $\displaystyle \lim_{x \to x_0}f(x) = \lim_{x \to x_0}g(x) = a$,则 $\displaystyle \lim_{x \to x_0}h(x) = a$。
\end{theorem}

\begin{theorem}
    设 $\displaystyle \lim_{x \to x_0}f(x)$ 和 $\displaystyle \lim_{x \to x_0}g(x)$ 存在。
    \begin{enumerate}
        \item 若 $\exists r > 0$,使得当 $x \in \check{N}_{r}(x_0)$ 时,有 $f(x) \leqslant g(x)$,则
              \[\lim_{x \to x_0}f(x) \leqslant \lim_{x \to x_0}g(x)\]
        \item 若 $\displaystyle \lim_{x \to x_0}f(x) < \lim_{x \to x_0}g(x)$,则存在 $\exists r > 0$,使得当 $x \in \check{N}_{r}(x_0)$ 时,
              \[f(x) < g(x)\]
    \end{enumerate}
\end{theorem}

\begin{theorem}{Cauchy 收敛原理}
    设函数 $f$ 在 $\check{N}_{r}(x_0)$ 有定义。则 $f(x)$ 有极限的充分必要条件是,对 $\forall \varepsilon > 0, \exists \delta > 0$,使得当 $x_1, x_2 \in \check{N}_{\delta}(x_0)$ 时,有
    \[|f(x_1) - f(x_2)| < \varepsilon\]
\end{theorem}

\begin{theorem}{复合函数的极限}
    设函数 $f$ 和 $g$。若 $\displaystyle \lim_{x \to x_0}f(x) = a, \lim_{t \to t_0}g(t) = x_0$,且在 $t_0$ 的某个邻域 $N_r(t_0)$ 内 $g(t) \neq x_0$,则
    \[\lim_{t \to t_0}f(g(t)) = a\]
\end{theorem}

\hfill

\begin{example}
    \[\lim_{x \to 0}\frac{\sin x}{x} = 1\]
\end{example}

\hfill

\begin{example}{(Dirichlet 函数)}
    定义函数 $D: \mathbb{R} \to \{ 0, 1 \}$ 为
    \[
        D(x) = \begin{cases}
            1,\quad x \in \mathbb{Q} \\
            0,\quad x \notin \mathbb{Q}
        \end{cases}
    \]

    则对 $\forall x_0 \in \mathbb{R}$,$\displaystyle \lim_{x \to x_0}D(x)$ 不存在。
\end{example}

\hfill

\begin{example}{(Riemann 函数)}
    定义函数 $R(x)$ 为
    \[
        R(x) = \begin{cases}
            1,\quad x = 0                                         \\
            \dfrac{1}{q},\quad x = \dfrac{p}{q}(q > 0, p \perp q) \\
            0,\quad x \notin \mathbb{Q}
        \end{cases}
    \]

    对任意 $x_0 \in \mathbb{R}$,$\displaystyle \lim_{x \to x_0}R(x) = 0$。
\end{example}

\begin{definition}
    设实数 $a$。
    \begin{enumerate}
        \item $\forall \varepsilon > 0, \exists A > 0$,当 $|x| > A$ 时,有
              \[|f(x) - a| < \varepsilon\]
              则称 $x$ 趋于无穷时,函数 $f$ 有极限 $a$。记为 $\displaystyle \lim_{x \to \infty}f(x) = a$。
        \item $\forall \varepsilon > 0, \exists A > 0$,当 $x < -A$ 时,有
              \[|f(x) - a| < \varepsilon\]
              则称 $x$ 趋于负无穷时,函数 $f$ 有极限 $a$。记为 $\displaystyle \lim_{x \to -\infty}f(x) = a$。
        \item $\forall \varepsilon > 0, \exists A > 0$,当 $x > A$ 时,有
              \[|f(x) - a| < \varepsilon\]
              则称 $x$ 趋于正无穷时,函数 $f$ 有极限 $a$。记为 $\displaystyle \lim_{x \to +\infty}f(x) = a$。
    \end{enumerate}
\end{definition}

\begin{theorem}
    $\displaystyle \lim_{x \to \infty}f(x) = a$ 当且仅当 $\displaystyle \lim_{x \to -\infty}f(x) = \lim_{x \to +\infty} = a$。
\end{theorem}

\hfill

\begin{example}
    \[\lim_{x \to \infty}\left(1 + \frac{1}{x}\right)^{x} = e\]
\end{example}

\hfill

\begin{definition}
    设 $x_0 \in \mathbb{R}$。$f$ 在 $\check{N}(x_0)$ 有定义。
    \begin{enumerate}
        \item 若对 $\forall A > 0$,$\exists \delta > 0$,使得当 $x \in \check{N}_{\delta}(x_0)$ 时,有 $|f(x)| > A$,则称 $x$ 趋于 $x_0$ 时,函数 $f$ 趋于无穷大,记为
              \[\lim_{x \to x_0}f(x) = \infty\]
        \item 若对 $\forall A > 0$,$\exists \delta > 0$,使得当 $x \in \check{N}_{\delta}(x_0)$ 时,有 $f(x) < -A$,则称 $x$ 趋于 $x_0$ 时,函数 $f$ 趋于负无穷大,记为
              \[\lim_{x \to x_{0}^{-}}f(x) = -\infty\]
        \item 若对 $\forall A > 0$,$\exists \delta > 0$,使得当 $x \in \check{N}_{\delta}(x_0)$ 时,有 $f(x) > A$,则称 $x$ 趋于 $x_0$ 时,函数 $f$ 趋于正无穷大,记为
              \[\lim_{x \to x_{0}^{+}}f(x) = +\infty\]
        \item 若对 $\forall A > 0$,$\exists \delta > 0$,使得当 $|x| > \delta$ 时,有 $|f(x)| > A$,则称 $x$ 趋于无穷时,函数 $f$ 趋于无穷大,记为
              \[\lim_{x \to \infty}f(x) = \infty\]
        \item 若对 $\forall A > 0$,$\exists \delta > 0$,使得当 $x < -\delta$ 时,有 $f(x) < -A$,则称 $x$ 趋于负无穷时,函数 $f$ 趋于负无穷大,记为
              \[\lim_{x \to -\infty}f(x) = -\infty\]
        \item 若对 $\forall A > 0$,$\exists \delta > 0$,使得当 $x > \delta$ 时,有 $f(x) > A$,则称 $x$ 趋于正无穷时,函数 $f$ 趋于正无穷大,记为
              \[\lim_{x \to +\infty}f(x) = +\infty\]
    \end{enumerate}
\end{definition}

\begin{definition}
    若某一极限过程中,$\lim f(x) =  0$,则称该过程中 $f$ 是一个无穷小。
\end{definition}

\begin{theorem}
    无穷大的倒数是无穷小;不取零值的无穷小的倒数是无穷大。
\end{theorem}

\begin{definition}
    设当 $x \to x_0$ 时,$f$ 和 $g$ 都是无穷小,且 $g$ 在 $\check{N}(x_0)$ 不取零值。
    \begin{enumerate}[itemsep=0.5em, topsep=0.5em]
        \item 若 $\displaystyle \lim_{x \to x_0}\frac{f(x)}{g(x)} = 0$,则称 $f$ 是比 $g$ 高阶的无穷小;
        \item 若 $\displaystyle \lim_{x \to x_0}\frac{f(x)}{g(x)} = a \neq 0$,则称 $f$ 是和 $g$ 同阶的无穷小;
        \item 若 $\displaystyle \lim_{x \to x_0}\frac{f(x)}{g(x)} = 1$,则称 $f$ 是和 $g$ 等价的无穷小,记为
              \[f \simeq g\quad (x \to x_0)\]
    \end{enumerate}
\end{definition}

当 $x \to x_0$ 时,取 $x - x_0$ 为 $1$ 阶无穷小。设当 $x \to x_0$ 时,$f$ 是无穷小。若 $f$ 与 $(x - x_0)^{\alpha}$ 为同阶无穷小,则称 $f$ 为 $\alpha$ 阶无穷小。

\begin{definition}
    设当 $x \to x_0$ 时,$f$ 和 $g$ 都是无穷大,且 $g$ 在 $\check{N}(x_0)$ 不取零值。
    \begin{enumerate}[itemsep=0.5em, topsep=0.5em]
        \item 若 $\displaystyle \lim_{x \to x_0}\frac{f(x)}{g(x)} = 0$,则称 $f$ 是比 $g$ 高阶的无穷大;
        \item 若 $\displaystyle \lim_{x \to x_0}\frac{f(x)}{g(x)} = a \neq 0$,则称 $f$ 是和 $g$ 同阶的无穷大;
        \item 若 $\displaystyle \lim_{x \to x_0}\frac{f(x)}{g(x)} = 1$,则称 $f$ 是和 $g$ 等价的无穷大,记为
              \[f \simeq g\quad (x \to x_0)\]
    \end{enumerate}
\end{definition}

当 $x \to x_0$ 时,取 $\dfrac{1}{x - x_0}$ 为 $1$ 阶无穷大。设当 $x \to x_0$ 时,$f$ 是无穷大。若 $f$ 与 $\left(\dfrac{1}{x - x_0}\right)^{\alpha}$ 为同阶无穷大,则称 $f$ 为 $\alpha$ 阶无穷大。

当 $x \to \infty$ 时,取 $|x|$ 为 $1$ 阶无穷大。设当 $x \to \infty$ 时,$f$ 是无穷大。若 $f$ 与 $|x|^{\alpha}$ 为同阶无穷大,则称 $f$ 为 $\alpha$ 阶无穷大。

\begin{remark}
    不是每个无穷小或无穷大都可以定阶。
\end{remark}

\begin{theorem}
    若当 $x \to x_0$($x_0$ 可以是 $\pm \infty$)时,$f, g$ 是等价的无穷小或无穷大,则
    \begin{enumerate}
        \item $\displaystyle \lim_{x \to x_0}f(x)h(x) = \lim_{x \to x_0}g(x)h(x)$;
        \item $\displaystyle \lim_{x \to x_0}\frac{f(x)}{h(x)} = \lim_{x \to x_0}\frac{g(x)}{h(x)}$。
    \end{enumerate}
\end{theorem}

\begin{definition}{Landau 记号}
    设函数 $f, g$ 在 $\check{N}(x_0)$ 有定义,且 $g(x) \neq 0$。
    \begin{enumerate}
        \item 若 $\exists M > 0$,使得当 $x \to x_0$ 时 $|f(x)| \leqslant M|g(x)|$ 成立,则记为
              \[f(x) = O(g(x))\quad (x \to x_0)\]
        \item 若当 $x \to x_0$ 时,$\dfrac{f(x)}{g(x)}$ 是无穷小,即 $\displaystyle \lim_{x \to x_0}\frac{f(x)}{g(x)} = 0$,则记为
              \[f(x) = o(g(x))\quad (x \to x_0)\]
    \end{enumerate}
    特别地,$f(x) = O(1)(x \to x_0)$ 表示函数 $f$ 有界, $f(x) = o(1)(x \to x_0)$ 表示函数 $f$ 是无穷小。
\end{definition}

\begin{remark}
    记号 $O(g(x))$ 和 $o(g(x))$ 只表示量的一种状态、一种类型,即左边的函数属于右边代表的类型。
\end{remark}

\begin{definition}
    设函数 $f$ 在 $\check{N}_{\delta}(x_0)$ 有定义。令
    \[E = \{ a \in \mathbb{R}_{\infty}: \exists x_n \in \check{N}_{\delta}(x_0), x_n \to x_0, \text{使得}\ f(x_n) \to a\}\]
    这是一个非空集合。则 $f$ 当 $x \to x_0$ 的上极限
    \[\sup E = \limsup_{x \to x_0}f(x)\]
    下极限
    \[\inf E = \liminf_{x \to x_0}f(x)\]
\end{definition}

\begin{remark}
    对其他极限过程($x \to x_{0}^{-}, x \to x_{0}^{+}, x \to -\infty, x \to +\infty, x \to \infty$)可类似地定义函数 $f$ 的上极限和下极限。
\end{remark}

\begin{theorem}
    设函数 $f$ 在 $\check{N}_{\delta}(x_0)$ 有定义。对于函数上极限 $\sup E$,有
    \begin{enumerate}
        \item $\sup E \in E$;
        \item 若 $y > \sup E$,则 $\exists \delta > 0$,使得当 $x \in \check{N}_{\delta}(x_0)$ 时 $f(x) < y$;
        \item $\sup E$ 是唯一满足前两条性质的数。
    \end{enumerate}
    对于函数下极限 $\inf E$,有
    \begin{enumerate}
        \item $\inf E \in E$;
        \item 若 $x < \inf E$,则 $\exists \delta > 0$,使得当 $x \in \check{N}_{\delta}(x_0)$ 时 $f(x) > x$;
        \item $\inf E$ 是唯一满足前两条性质的数。
    \end{enumerate}
\end{theorem}

\begin{theorem}
    设函数 $f, g$ 在 $\check{N}_{\delta}(x_0)$ 有定义。则
    \begin{enumerate}[itemsep=0.5em]
        \item $\displaystyle \liminf_{x \to x_0}f(x) \leqslant \limsup_{x \to x_0}f(x)$;
        \item $\displaystyle \lim_{x \to x_0}f(x) = a$ 当且仅当 $\displaystyle \liminf_{x \to x_0}f(x) = \limsup_{x \to x_0}f(x) = a$;
        \item 若当 $x \in \check{N}_{\delta}(x_0)$ 时,有 $f(x) \leqslant g(x)$,则
              \begin{gather*}
                  \liminf_{x \to x_0}f(x) \leqslant \liminf_{x \to x_0}g(x) \\
                  \limsup_{x \to x_0}f(x) \leqslant \limsup_{x \to x_0}g(x)
              \end{gather*}
    \end{enumerate}
\end{theorem}

%——————————————————————————————————%

\section{连续函数}






%——————————————————————————————————%