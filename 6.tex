\chapter{多元函数微分}

%——————————————————————————————————%

\section{多元函数的极限}

\begin{definition}
    设 $D \subseteq \mathbb{R}^n$,则映射 $f: D \to \mathbb{R}$ 称为 $n$ 元函数,其中 $D$ 称为函数 $f$ 的定义域,$f(D) \subseteq \mathbb{R}$ 称为 $f$ 的值域。

    设点 $\bm{x} \in D$,记为 $\bm{x} = (x_1, x_2, \ldots, x_n)$。则 $f$ 在点 $\bm{x}$ 取值为 $f(\bm{x})$,也记为 $f(x_1, x_2, \ldots, x_n)$。变元 $x_{i}(i = 1, 2, \ldots, n)$ 称为 $f$ 的自变量。
\end{definition}

\begin{definition}
    设 $D \subseteq \mathbb{R}^n, f: D \to \mathbb{R}$,点 $\bm{a} \in \mathbb{R}^n$ 是 $D$ 的凝聚点(即 $\bm{a} \in D'$),$l \in \mathbb{R}$。若对 $\forall \varepsilon > 0, \exists \delta > 0$,当 $\bm{x} \in D$ 且 $0 < \Vert \bm{x} - \bm{a} \Vert < \delta$ 时,有
    \[
        \vert f(\bm{x}) - l \vert < \varepsilon
    \]
    则称函数 $f$ 在点 $\bm{a}$ 处有极限 $l$。或说当 $\bm{x}$ 趋向于 $\bm{a}$ 时,$f(\bm{x})$ 趋向于 $l$,记为
    \[
        \lim_{\bm{x} \to \bm{a}} f(\bm{x}) = l
    \]
    也可简写为
    \[
        f(\bm{x}) \to l \quad (\bm{x} \to \bm{a})
    \]
\end{definition}

\begin{theorem}
    设 $D \subseteq \mathbb{R}^n, f: D \to \mathbb{R}$,点 $\bm{a} \in D'$。函数极限
    \[
        \lim_{\bm{x} \to \bm{a}} f(\bm{x}) = l
    \]
    的充分必要条件是,对任何点列 $\{\bm{x}_{i}\} \subseteq D, \bm{x_{i}} \neq \bm{a}(i = 1, 2, \ldots)$ 且 $\bm{x}_{i} \to \bm{a}(i \to \infty)$,数列极限
    \[
        \lim_{i \to \infty} f(\bm{x}_{i}) = l
    \]
\end{theorem}

\begin{theorem}
    设 $D \subseteq \mathbb{R}^n, f, g: D \to \mathbb{R}$,点 $\bm{a} \in D'$。若 $f, g$ 存在有限的极限
    \[
        \lim_{\bm{x} \to \bm{a}} f(\bm{x}) = l,\quad
        \lim_{\bm{x} \to \bm{a}} g(\bm{x}) = m
    \]
    则
    \begin{enumerate}
        \item $\displaystyle \lim_{\bm{x} \to \bm{a}} (f \pm g)(\bm{x}) = l \pm m$;
        \item $\displaystyle \lim_{\bm{x} \to \bm{a}} (fg)(\bm{x}) = lm$;
        \item $\displaystyle \lim_{\bm{x} \to \bm{a}} \left(\frac{f}{g}\right)(\bm{x}) = \frac{l}{m}$,其中 $m \neq 0$。
    \end{enumerate}
\end{theorem}

\begin{theorem}
    设函数 $f$ 在以 $\bm{a} \in \mathbb{R}^n$ 为球心、$r$ 为半径的某个空心球 $B_r(\check{a})$ 上有定义,且 $\displaystyle \lim_{i \to \infty} f(\bm{x}) = l$;一元函数 $\varphi$ 在以 $l$ 为球心的空心球 $U = \{t: 0 < \vert t - l \vert < \delta\}$ 上有定义,且 $\displaystyle \lim_{t \to l} \varphi(t) = m$。再设
    \[
        f(B_r(\check{a})) \subseteq U
    \]
    则有
    \[
        \lim_{\bm{x} \to \bm{a}} \varphi(f(\bm{x})) = m
    \]
\end{theorem}

\begin{theorem}{Cauchy 收敛原理}
    设 $D \subseteq \mathbb{R}^n, f: D \to \mathbb{R}$,点 $\bm{a} \in D'$。极限 $\displaystyle \lim_{\bm{x} \to \bm{a}} f(\bm{x})$ 存在的充分必要条件是,对 $\forall \varepsilon > 0, \exists \delta > 0$,使得当 $\bm{x}', \bm{x}'' \in D$ 且
    \[
        0 < \Vert \bm{x}' - \bm{a} \Vert < \delta, \quad 0 < \Vert \bm{x}'' - \bm{a} \Vert < \delta
    \]
    同时成立,一定有 $\vert f(\bm{x}') - f(\bm{x}'') \vert < \varepsilon$。
\end{theorem}

%——————————————————————————————————%

\section{多元连续函数}

\begin{definition}
    设 $D \subseteq \mathbb{R}^n, f: D \to \mathbb{R}, \bm{a} \in D$。若对 $\forall \varepsilon > 0, \exists \delta > 0$,使得当 $\bm{x} \in D \cap B_{\delta}(\bm{a})$ 时,有
    \[
        \vert f(\bm{x}) - f(\bm{a}) \vert < \varepsilon
    \]
    则称函数 $f$ 在点 $\bm{a}$ 处连续。$\bm{a}$ 称为 $f$ 的连续点。$D$ 中 $f$ 的非连续点称为 $f$ 的间断点。

    若 $f$ 在 $D$ 中每个点上都连续,则称 $f$ 在 $D$ 上连续。
\end{definition}

\begin{remark}
    $f$ 在 $\bm{a}$ 处连续必须在点 $\bm{a}$ 处有定义。
\end{remark}

\begin{theorem}{多元连续函数的四则运算}

\end{theorem}

\begin{theorem}{多元连续函数的复合}

\end{theorem}

\begin{definition}
    设 $D \subseteq \mathbb{R}^n, f : D \to \mathbb{R}$。若对 $\forall \varepsilon > 0$,总存在 $\delta > 0$,使得当 $\bm{x}, \bm{y} \in D$ 且 $\Vert \bm{x} - \bm{y} \Vert < \delta$ 时,有 $\vert f(\bm{x}) - f(\bm{y}) \vert < \varepsilon$,则称 $f$ 在 $D$ 上一致连续。
\end{definition}

\begin{remark}
    对有限点集 $D \subseteq \mathbb{R}^n$,任何函数 $f: D \to \mathbb{R}$ 在 $D$ 上都是一致连续的。特别地,常值函数在任何点集 $D$ 上都是一致连续的。
\end{remark}

\begin{theorem}
    设 $D \subseteq \mathbb{R}^n, f: D \to \mathbb{R}$,且 $f$ 在 $D$ 上连续。若 $D$ 是紧致集,则 $f$ 在 $D$ 上一致连续。
\end{theorem}

\begin{theorem}
    设 $D \subseteq \mathbb{R}^n, f: D \to \mathbb{R}$,且 $f$ 在 $D$ 上连续。若 $D$ 是紧致集,则 $f$ 的值域 $f(D)$ 也是紧致集。
\end{theorem}

\begin{theorem}
    设 $D \subseteq \mathbb{R}^n, f: D \to \mathbb{R}$,且 $f$ 在 $D$ 上连续。若 $D$ 是紧致集,则 $f$ 在 $D$ 上能取到它的最小值和最大值。
\end{theorem}

\begin{theorem}
    设 $D \subseteq \mathbb{R}^n$ 是连通集, 函数 $f: D \to \mathbb{R}$ 在 $D$ 上连续,则 $f(D)$ 是 $\mathbb{R}$ 中的连通集。
\end{theorem}

\begin{theorem}{介值定理}
    设 $D \subseteq \mathbb{R}^n$ 是连通集, 函数 $f: D \to \mathbb{R}$ 在 $D$ 上连续。若有 $\bm{a}, \bm{b} \in D$ 和 $r \in \mathbb{R}$ 使得
    \[
        f(\bm{a}) < r < f(\bm{b})
    \]
    则 $\exists c \in D$,使得 $f(\bm{c}) = r$。
\end{theorem}

\begin{definition}
    设 $D$ 是 $\mathbb{R}^n$ 中的凸区域。若对 $\forall \bm{x}, \bm{y} \in D$ 及 $\forall \lambda \in [0, 1]$,有
    \[
        f(\lambda \bm{x} + (1 - \lambda)\bm{y}) \leqslant \lambda f(\bm{x}) + (1 - \lambda)f(\bm{y})
    \]
    则称 $f$ 为 $D$ 上的凸函数。
\end{definition}

\begin{proposition}
    凸区域 $D$ 上的凸函数必定是 $D$ 上的连续函数。
\end{proposition}


%——————————————————————————————————%

\section{连续映射}

\begin{definition}
    设 $D \subseteq \mathbb{R}^n$,若有函数 $\bm{f}: D \to \mathbb{R}^m$,$\bm{f}$ 的值是 $m$ 维欧式空间中的点,即一个 $m$ 维向量,则称函数 $\bm{f}$ 是在 $\mathbb{R}^m$ 中取值的向量值函数。
    \[
        \bm{f} = (f_1, f_2, \ldots, f_n)
    \]
    其中 $f_{i}: D \to \mathbb{R}(i = 1, 2, \ldots, m)$ 称为 $\bm{f}$ 的第 $i$ 个分量函数。

    设 $\bm{y} = \bm{f}(\bm{x})(x \in D)$,则 $\bm{y} = (y_1, y_2, \ldots, y_m)$,$\bm{x} = (x_1, x_2, \ldots, x_n)$,给定 $\bm{f}$ 相当于给定 $m$ 个 $n$ 元函数:
    \begin{equation*}
        \begin{cases}
            y_1 = f_{1}(x_1, x_2, \ldots, x_n), \\
            y_2 = f_{1}(x_1, x_2, \ldots, x_n), \\
            \cdots                              \\
            y_m = f_{1}(x_1, x_2, \ldots, x_n)
        \end{cases} (x_1, x_2, \ldots, x_n) \in D \subseteq \mathbb{R}^n
    \end{equation*}
\end{definition}

\begin{definition}
    设 $D \subseteq \mathbb{R}^n, \bm{f}: D \to \mathbb{R}^m$,又设 $\bm{a} \in D', \bm{p} \in \mathbb{R}^m$。若对 $\forall \varepsilon > 0, \exists \delta > 0$,使得当 $\bm{x} \in D$ 且 $0 < \Vert \bm{x} - \bm{a} \Vert < \delta$ 时,有
    \[
        \Vert\bm{f}(\bm{x}) - \bm{p} \Vert < \varepsilon
    \]
    则称映射 $\bm{f}$ 在点 $\bm{a}$ 处有极限 $\bm{p}$,记为
    \[
        \lim_{\bm{x} \to \bm{a}}\bm{f}(\bm{x}) = \bm{p}
    \]
    简记为
    $\bm{f}(\bm{x}) \to \bm{p} \quad (\bm{x} \to \bm{a})$
\end{definition}

\begin{remark}
    该定义的几何语言描述:对任意给定的球 $B_{\varepsilon}(\bm{p}) \subseteq \mathbb{R}^m$,必定存在一个球 $B_{\delta}(\bm{a}) \subseteq \mathbb{R}^n$,当 $D$ 中的点 $\bm{x}$ 在空心球 $B_{\delta}(\check{a})$ 中时,它的像必定在球 $B_{\varepsilon}(\bm{p})$ 中。
\end{remark}

\begin{theorem}
    设 $D \subseteq \mathbb{R}^n, \bm{f}: D \to \mathbb{R}^m, \bm{a} \in D', \bm{p} = (p_1, p_2, \ldots, p_m) \in \mathbb{R}^m, \bm{f} = (f_1, f_2, \ldots, f_m)$,则
    \[
        \lim_{\bm{x} \to \bm{a}}\bm{f}(\bm{x}) = \bm{p}
    \]
    当且仅当
    \[
        \lim_{\bm{x} \to \bm{a}}f_{i}(\bm{x}) = p_{i} \quad (i = 1, 2, \ldots, m)
    \]
\end{theorem}

\begin{theorem}
    设 $D \subseteq \mathbb{R}^n, \bm{a} \in D'$,又设 $f, g: D \to \mathbb{R}^m$,且
    \[
        \lim_{\bm{x} \to \bm{a}}\bm{f}(\bm{x}) = \bm{p}, \quad \lim_{\bm{x} \to \bm{a}}\bm{g}(\bm{x}) = \bm{q}
    \]
    则有
    \begin{enumerate}
        \item 对 $\forall \lambda \in \mathbb{R}$,$\displaystyle \lim_{\bm{x} \to \bm{a}}(\lambda \bm{f}(\bm{x})) = \lambda \bm{p}$;
        \item $\displaystyle \lim_{\bm{x} \to \bm{a}}(\bm{f}(\bm{x}) + \bm{g}(\bm{x})) = \bm{p} + \bm{q}$。
    \end{enumerate}
\end{theorem}

\begin{definition}
    设点集 $D \subseteq \mathbb{R}^n, f: D \to \mathbb{R}^m, \bm{a} \in D$。若对 $\forall \varepsilon > 0, \exists \delta > 0$,使得当 $\bm{x} \in D \cap B_{\delta}(\bm{a})$ 时,有 $\bm{f}(\bm{x}) \in B_{\varepsilon}(\bm{f}(\bm{a}))$,则称映射 $\bm{f}$ 在点 $\bm{a}$ 处连续。

    当 $\bm{f}$ 在 $D$ 中的每一点都连续时,则称映射 $f$ 在 $D$ 上连续。
\end{definition}

\begin{remark}
    映射 $\bm{f}$ 在 $D$ 中的一点 $\bm{a}$ 处连续,必须且只需 $f$ 的每一个分量函数在点 $\bm{a}$ 处连续。
\end{remark}

\begin{theorem}
    设 $D$ 是 $\mathbb{R}^n$ 中的开集,$\bm{f}: D \to \mathbb{R}^m$,$\bm{f}$ 在 $D$ 上连续的充分必要条件是,对 $\mathbb{R}^m$ 中的任意开集 $G$,$f^{-1}(G)$ 是 $\mathbb{R}^n$ 中的开集。其中
    \[
        f^{-1}(G) = \{\bm{x} \in D: \bm{f}(\bm{x}) \in G\}
    \]
\end{theorem}

\begin{definition}
    设 $D$ 是 $\mathbb{R}^n$ 中的开集,$\bm{f}: D \to \mathbb{R}^m$。若对 $\forall \varepsilon > 0, \exists \delta > 0$,使得当 $\bm{x}, \bm{y} \in D$ 且 $\Vert \bm{x} - \bm{y} \Vert < \delta$ 时,均有
    \[
        \Vert \bm{f}(\bm{x}) - \bm{f}(\bm{y}) \Vert < \varepsilon
    \]
    则称映射 $\bm{f}$ 在 $D$ 上一致连续。
\end{definition}

\begin{remark}
  常值函数在其定义域上是一致连续的。若 $D \subseteq \mathbb{R}^n$ 是有限点集,则任何映射 $\bm{f}: D \to \mathbb{R}^m$ 在 $D$ 上一致连续。
\end{remark}

\begin{remark}
  映射 $\bm{f}$ 在 $D$ 上一致连续当且仅当 $\bm{f}$ 的每个分量函数在 $D$ 上一致连续。
\end{remark}

\begin{theorem}
    设 $D \subseteq \mathbb{R}^n, f: D \to \mathbb{R}^m$ 是 $D$ 上的连续映射。若 $D$ 是紧致集,则 $f$ 在 $D$ 上是一致连续的。
\end{theorem}

\begin{theorem}
    设 $D \subseteq \mathbb{R}^n, f: D \to \mathbb{R}^m$ 是 $D$ 上的连续映射。若 $D$ 是 $\mathbb{R}^n$ 中的连通集,则 $\bm{f}(D)$ 是 $\mathbb{R}^m$ 中的连通集。
\end{theorem}

\begin{theorem}
    设 $D \subseteq \mathbb{R}^n, f: D \to \mathbb{R}^m$ 是 $D$ 上的连续映射。若 $D$ 是 $\mathbb{R}^n$ 中的紧致集,则 $\bm{f}(D)$ 是 $\mathbb{R}^m$ 中的紧致集。
\end{theorem}

%——————————————————————————————————%

\section{方向导数和偏导数}

\begin{definition}
  设开集 $D \subseteq \mathbb{R}^n, f: D \to \mathbb{R}$。$\mathbb{R}^n$ 中的任一单位向量 $\bm{u}$($\bm{u}$ 满足 $\Vert \bm{u} \Vert = 1$)称为一个方向。给定一个点 $\bm{x}_0 \in D$ 和一个方向 $\bm{u}$,通过点 $\bm{x}_0$ 和 $\bm{x}_0 + \bm{u}$ 的直线称为过点 $\bm{x}_0$ 的具有方向 $\bm{u}$ 的直线,即由点集
  \[
      \{\bm{x}: \bm{x} = \bm{x}_0 + t \bm{u}, t \in \mathbb{R}\}
  \]
  组成的直线。
\end{definition}

\begin{definition}
  设开集 $D \subseteq \mathbb{R}^n, f: D \to \mathbb{R}$,$\bm{u}$ 是一个方向,$\bm{x}_0 \in D$。若极限
  \[
      \lim_{t \to 0}\frac{f(\bm{x}_0 + t \bm{u}) - f(\bm{x}_0)}{t}
  \]
  存在且有限,则称该极限是函数 $f$ 在点 $\bm{x}_0$ 处沿方向 $\bm{u}$ 的方向导数,记为 $\dfrac{\partial f}{\partial \bm{u}}(\bm{x}_0)$
\end{definition}

\begin{definition}
  函数 $f$ 在点 $\bm{x}_0$ 处沿方向 $\bm{e}_i$ 的方向导数为 $f$ 在 $\bm{x}_0$ 处的第 $i$ 个一阶偏导数,记为
  \[
      \frac{\partial f}{\partial x_i}(\bm{x}_0) \quad \text{或} \quad \mathrm{D}_{i}f(\bm{x}_0)
  \]
  并称 $\mathrm{D}_i = \dfrac{\partial}{\partial x_i}$ 为第 $i$ 个偏微分算子($i = 1, 2, \ldots, n$)。
\end{definition}

\begin{remark}
  一般地,计算 $\mathrm{D}_{i}f(\bm{x}_0)$ 时,只需对第 $i$ 个变数求导,同时将其余变数视为常数($i = 1, 2, \ldots, n$)。
\end{remark}

\begin{definition}
  $\mathbb{R}^3$ 中的点集
  \[
      G(f) = \{(x, y, f(x, y)): (x, y) \in D\}
  \]
  称为函数 $f$ 的图像。
\end{definition}



%——————————————————————————————————%

\section{}






%——————————————————————————————————%

\section{}






%——————————————————————————————————%

\section{}






%——————————————————————————————————%

\section{}






%——————————————————————————————————%

\section{}






%——————————————————————————————————%

\section{}






%——————————————————————————————————%

\section{}






%——————————————————————————————————%

\section{}






